\documentclass[12pt]{article}
\usepackage{parskip} % no auto indent
\usepackage{graphicx} % graphics

% My marco file
\usepackage{mymarco}

% Begin page style
\usepackage[margin=1.0in]{geometry}
\usepackage{fancyhdr}
\pagestyle{fancy}
\setlength{\headheight}{15pt}
\rhead{36123040 Shawn Wu} % Define header
\lhead{MATH 321 HW 02} % Put assignment #
\cfoot{\thepage}

\begin{document}

% Problem 1
\begin{fproof}[1(a)]
    We fix \(c \in (a,b)\).
    We need to show that LHS \(\leq\) RHS and LHS \(\geq\) RHS.
    Let partition \(P\) on \([a,b]\) be arbitrary.
    Then let \(P^*\) by its refinement by adding the point \(c\).
    Then split \(P^*\) into \(P_1\) and \(P_2\), where \(P_1\) is a partition for \([a,c]\) and \(P_2\) is a partition for \([c,b]\).
    Then,
    \begin{align*}
        U(P, f|_{[a,b]}) &\geq U(P^*, f|_{[a,b]}) = U(P_1, f|_{[a,c]}) + U(P_2, f|_{[c,b]}) \geq \inf_P U(P, f|_{[a,c]}) + \inf_P U(P, f|_{[c,b]})\\
        & = \upint{a}{c} f \dalf + \upint{c}{b} f\dalf.
    \end{align*}
    This means that \(\upint{a}{c} f \dalf + \upint{c}{b} f\dalf\) is a lower bound for \(U(P, f|_{[a,b]})\) over all partitions of \([a,b]\), so it must be smaller than the greatest lower bound. Therefore,
    \begin{align*}
        \upint{a}{b} f \dalf = \inf_{P} U(P, f|_{[a,b]}) \geq \upint{a}{c} f \dalf + \upint{c}{b} f\dalf.
    \end{align*}
    Hence, LHS \(\geq\) RHS.

    Let \(\varepsilon > 0\) be given.
    By the property of infinum, there must be partition \(P_1\) of \([a,c]\) and partition \(P_2\) of \([c,b]\) such that
    \begin{align*}
        U(P_1, f|_{[a,c]}) < \upint{a}{c} f \dalf + \frac{\varepsilon}{2}, \text{ and } U(P_2, f|_{[c,b]}) < \upint{c}{b} f \dalf + \frac{\varepsilon}{2}.
    \end{align*}
    Note that \(P_1 \cup P_2\) is a partition for \([a,b]\).
    Therefore,
    \begin{align*}
        \upint{a}{b} f \dalf \leq U(P_1 \cup P_2, f|_{[a,b]})  = U(P_1, f|_{[a,c]}) + U(P_2, f|_{[c,b]}) < \upint{a}{c} f \dalf + \upint{c}{b} f \dalf + \varepsilon.
    \end{align*}
    Since \(\varepsilon\) is arbitrary,
    \begin{align*}
        \upint{a}{b} f \dalf \leq \upint{a}{c} f \dalf + \upint{c}{b} f \dalf.
    \end{align*}
    Hence, LHS \(\leq\) RHS.
\end{fproof}

\begin{fproof}[1(b)]
    First note that over any sub-interval \([x_{i-1},x_i] \subseteq [a,b]\), 
    \begin{align*}
        \sup (f+g) \leq \sup f + \sup g. \tag{\(\ast\)}
    \end{align*}
    To see this, let \(\varepsilon > 0\) be given.
    Then there exists \(x \in [x_{i-1}, x_i]\) s.t. \(\sup(f + g) - \varepsilon < (f+g)(x)\).
    Therefore,
    \begin{align*}
        \sup(f + g) - \varepsilon < f(x) + g(x) \leq \sup f + \sup g.
    \end{align*}
    And since \(\varepsilon\) is arbitrary, \(\sup(f+g) \leq \sup f + \sup g\).

    Now we are ready to prove the inequality in question.
    Again let \(\varepsilon > 0\) be given. 
    Then there exist partitions \(P_1, P_2\) of \([a,b]\) s.t.
    \begin{align*}
        U(P_1, f) < \upint{a}{b} f \dalf + \frac{\varepsilon}{2}, \text{ and } U(P_2, g) < \upint{a}{b} g \dalf + \frac{\varepsilon}{2}. 
    \end{align*}
    Let \(P^* = P_1 \cup U_2\).
    By \((\ast)\), \(U(P^*, f+g) \leq U(P^*, f) + U(P^*, g)\).
    Also, by property of refinement, \(U(P^*,f) \leq U(P_1, f)\) and \(U(P^*, g) \leq U(P_2, g)\).
    Hence,
    \begin{align*}
        \upint{a}{b} f \dalf \leq U(P^*, f+g) \leq U(P^*,f) + U(P^*,g) < \upint{a}{b} f \dalf + \upint{a}{b} g \dalf + \varepsilon.
    \end{align*}
    And since \(\varepsilon\) is arbitrary, LHS \(\leq\) RHS as desired.

\end{fproof}

\begin{fproof}[1(c)]
    Let \([a,b] = [0,1]\)
    Consider \(\alpha(x) = x\) on \([0,1]\). And let
    \[
    f(x) = \begin{cases}
        -1 &\text{ if } x \in \mathbb{Q} \cap [0,1],\\
        1 &\text{ otherwise,}
    \end{cases}
    \text{ and }
    g(x) = \begin{cases}
        1 &\text{ if } x \in \mathbb{Q} \cap [0,1],\\
        -1 &\text{ otherwise.}
    \end{cases}
    \]
    Then \(f+g = 0\) on \([a,b]\).
    Since for any sub-intervals \([x_{i-1}, x_i]\) of \([0,1]\) contain both rational and irrational points, \(M_i = 1\) for both \(f\) and \(g\) for all \(i\).
    Therefore, 
    \begin{align*}
        0 = \upint{0}{1}(f+g) \dx < \upint{0}{1} f \dx + \upint{0}{1} g \dx = 1 + 1 = 2.
    \end{align*}
\end{fproof}
\newpage

% Problem 2
\begin{fproof}[2]
    For the sake of contradiction, suppose that there is indeed a bounded \(\rho \in \mathcal{R}[-1,1]\) s.t. \(\int_{-1}^{1} f(x) \rho(x) \dx = f(0)\) for any \(f \in C[-1,1]\).
    Consider the following sequence of continuous functions on \([-1,1]\),
\[
        f_n = \begin{cases}
            n^2x + n & \text{ if } x \in [-\frac{1}{n}, 0],\\
            -n^2x + n & \text{ if } x \in [0, \frac{1}{n}],\\
            0 & \text{ otherwise.}
        \end{cases}
\]
    Note that each \(f_n\) forms a thin triangle of area of exactly \(1\) on \([-1,1]\) where \(f_n(0) = n\).

    Now let \(M \in \mathbb{R}\) be such that \(\sup_{x \in [-1,1]} \abs{\rho(x)} \leq M\). Such \(M\) exists as \(\rho\) is bounded on \([-1,1]\).
    Then, for all \(n \in \mathbb{N}\),
    \begin{align*}
        n = f_n(0) = \int_{-1}^{1} f(x) \rho(x) \dx \leq \int_{-1}^{1} f(x) M \dx = M \int_{-1}^{1} f(x) \dx = M.
    \end{align*}
    This is a contradiction since \(\mathbb{N}\) is not bounded above.
\end{fproof}
\newpage

% Problem 3
\begin{fproof}[3]
  First, note that the mass function of length \(x\) from the origin is given by
\[
    M(x) = \begin{cases}
        \mu x & \text{ if } x \in [0, s_1),\\
        \mu x + m_1 & \text{ if } x \in [s_1, s_2),\\
        \mu x + m_1 + m_2 & \text{ if } x \in [s_2, 1].
    \end{cases}
\]
Or simply \(M(x) = M_0(x) + M_1(x) + M_2(x)\), where
\[
    M_0(x) = \mu x, M_1(x) = m_1I(x-s_1), M_2 = m_2I(x-s_2), \text{ where } I(t) = 
    \begin{cases}
        0 & \text{ if } t < 0,\\
        1 & \text{ if } t \geq 0.
    \end{cases}
\]
  We can thus define \(M(a,b) = M(b) - M(a)\).
  Therefore, it remains to find \(\rho(t)\) on \([0,1]\) such that \(M(x) = \int_{0}^{x} \drho(t)\).

  Consider,
  \[
    \rho(t) = \mu t + m_1I(t-s_1) + m_2I(t-s_2).
  \]
  Therefore, by the linearity of Stieltjes integrals,
  \begin{align*}
    M(x) = \int_{0}^{x} \drho(t) = \mu \int_{0}^{x} \dt + m_1 \int_{0}^{x} \diff(I(t-s_1)) + m_2 \int_{0}^{x} \diff(I(t-s_2)).
  \end{align*}
  And note that by Rudin Theorem 6.15,
  \begin{align*}
    m_1 \int_{0}^{x} \diff(I(t-s_1)) = M_1, \text{ and }  m_2 \int_{0}^{x} \diff(I(t-s_2)) = M_2.
  \end{align*}
  Also note that
  \begin{align*}
    \mu \int_{0}^{x} \dt = \mu x = M_0(x).
  \end{align*}
  Therefore, \(M(x) = M_0(x) + M_1(x) + M_2(x)\) as desired.
\end{fproof}
\newpage

% Problem 4
\begin{fproof}[4(a)]
    Let \(\varepsilon > 0\) be arbitrary.
    We show that there exists a natural \(N\), such that for all \(n > N\), \(\abs{U(P_n, f) - L(P_n, f) - 0} < \varepsilon\), or simply \(U(P_n, f) - L(P_n, f) < \varepsilon\) since \(U(P_n, f) \geq L(P_n, f)\) always true.
  
    Since \(f \in \mathcal{R}[a,b]\), then there exists a partition \(Q = \{q_0, q_1, \cdots, q_k\}\) of \([a,b]\) where \(U(Q, f) - L(Q,f) < \varepsilon/2\).
    Let \(l_{min}\) be the minimum length of the closed intervals \([q_{j-1}, q_j]\) where \(j = 1,\cdots,k\).
  
    We also know that \(f\) has to be bounded. Therefore, let \(M \geq \abs{f(x)}\) for all \(x \in [a,b]\).
  
    Let \(P_n = \{x_0, x_1, \cdots, x_n\}\).
    We choose \(N\) so large that \(d_N = \frac{b-a}{N} < l_{min}\).
    Then by generalized pigeonhole principle, each interval \([x_{i-1}, x_i]\) from \(P_n\) contains at most one point from Q. And note that for all \(n > N\), such property still holds.
    And make \(N\) so large that \(d_N < \frac{\varepsilon/2}{2M(k-1)}\).
  
    Consider the partition \(Q \cup P_n\), which is a refinement of both \(Q\) and \(P_n\). 
    Therefore, we know \[U(Q \cup P_n, f) - L(Q \cup P_n, f) \leq U(P_n, f) - L(P_n, f),\] and \[U(Q \cup P_n, f) - L(Q \cup P_n, f) \leq U(Q, f) - L(Q, f) < \varepsilon/2.\]
    Now look more closely, we first see that there are \(k-1\) points from \(Q\) that are not the end points.
    And therefore, there are exactly \(k-1\) intervals in the form of \([x_{i-1}, x_i]\) from \(P_n\) (\(n > N\)) that contains exactly one point of \(Q\), where \(i \in \{2, 3, \cdots, n-1\}\).
  
    Let \(D_n = \brac{U(P_n,f) - L(P_n, f)} - \brac{U(Q \cup P_n, f) - L(Q \cup P_n, f)}\).
    And suppose \([x_{i-1}, x_i]\) contains exactly one \(q_j\) where \(j = 2, \cdots, k-2\), define \[\Lambda_j = \pare{\sup_{x \in [x_{i-1}, q_j]}f(x) - \inf_{x \in [x_{i-1}, q_j]}f(x)}\abs{x_{i-1} - q_j} + \pare{\sup_{x \in [q_j, x_i]}f(x) + \inf_{x \in [q_j, x_i]}f(x)}\abs{q_j - x_i}.\]
    Notice that \(\Lambda_j \geq 0\) always.
    
    Therefore, we have
    \begin{align*}
      D_N &= \brac{U(P_N,f) - L(P_N, f)} - \brac{U(Q \cup P_N, f) - L(Q \cup P_N, f)}\\
      & = \sum_{[x_{i-1}, x_i] \cap \{q_1, \cdots, q_j, \cdots, q_{k-1}\} \neq \varnothing} (M_i - m_i) \Delta x_i - \Lambda_j\\
      & \leq \sum_{[x_{i-1}, x_i] \cap \{q_1, \cdots, q_j, \cdots, q_{k-1}\} \neq \varnothing} (M_i - m_i) \Delta x_i \\
      & = d_N \cdot \pare{\sum_{[x_{i-1}, x_i] \cap \{q_1, \cdots, q_j, \cdots, q_{k-1}\} \neq \varnothing} (M_i - m_i)}\\
      & \leq d_N \cdot \pare{\sum_{[x_{i-1}, x_i] \cap \{q_1, \cdots, q_j, \cdots, q_{k-1}\} \neq \varnothing} 2M}\\
      & = d_N \cdot 2M(k-1)\\
      & < \varepsilon/2.
    \end{align*}
    And note that as for all \(n > N\), we will still have \(D_n \leq d_n \cdot 2M(k-1) < \varepsilon/2\).
  
    Therefore, we see that for all \(n > N\),
    \begin{align*}
      U(P_n, f) - L(P_n,f) = U(Q \cup P_n, f) - L(Q \cup P_n, f) + D_n < \varepsilon/2 + \varepsilon/2 = \varepsilon.
    \end{align*}
\end{fproof}

\begin{fproof}[4(b)]
    We give a counter-example.
    Let \([a,b] = [0,1]\). Let \(\phi\) be the Golden Ratio (\(\approx 0.618\)), a irrational number.
    Define
    \[
    f(x) = \begin{cases}
      0, \text{ if } x < \phi\\
      1, \text{ if } x \geq \phi,
    \end{cases}
    \]
    and 
    \[
    \alpha(x) = \begin{cases}
      0, \text{ if } x \leq \phi\\
      1, \text{ if } x > \phi.
    \end{cases}
    \]
  
    We can see that \(f \in \mathcal{R}(\alpha)\) on \([0,1]\).
    Note that if we choose partition \(P = [0, \phi, 1]\), then 
    \begin{align*}
      U(P,f) - L(P,f) &= (M_1 - m_1)(\alpha(\phi) - \alpha(0)) + (M_2 - m_2)(\alpha(1) - \alpha(\phi)) \\
      & = (M_2 - m_2)(\alpha(1) - \alpha(\phi)) \tag{as \(\alpha(1) = \alpha(\phi)\)}\\
      & = 0 \tag{as \(M_2 = m_2\)}.
    \end{align*}
    This means that for all \(\varepsilon > 0\), there exists a partition \(P\) such that \(U(P,f) - L(P,f) < \varepsilon\). Thus, by Rudin Theorem 6.6, we have \(f \in \mathcal{R}(\alpha)\) on \([0,1]\).
  
    Now note that for all \(n \in \N\), all points in \(P_n\) remain to be rational.
    Therefore, no matter how large \(n\) is, \(\phi \neq p\) for all \(p \in P_n\).
    And there is exactly one interval \([x_{i-1}, x_i]\) from \(P_n\) that contains \(\phi\).
  
    Therefore, for any \(n \in \N\),
    \begin{align*}
      U(P_n, f, \alpha) - L(P_n, f, \alpha) &= \sum_{i=1}^{n} (M_i - m_i) \Delta \alpha_i\\
      &= \sum_{\phi \in [x_{i-1}, x_i]} (M_i - m_i)\Delta \alpha_i\\
      &= (1 - 0) \cdot (1-0)\\
      & = 1
    \end{align*}
    Therefore, the limit cannot go to zero.
\end{fproof}
\end{document}