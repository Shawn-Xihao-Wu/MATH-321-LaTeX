\documentclass[12pt]{article}
\usepackage{parskip} % no auto indent

% My marco file
\usepackage{mymarco}

% Begin page style
\usepackage[margin=1.0in]{geometry}
\usepackage{fancyhdr}
\pagestyle{fancy}
\setlength{\headheight}{15pt}
\rhead{36123040 Shawn Wu} % Define header
\lhead{MATH 321 HW 07} % Put assignment #
\cfoot{\thepage}

\begin{document}

% Problem 1
\begin{fproof}[1(a)]
 Fix any \(x\) where \(\abs{x} < 1\).
 First if \(x = 0\). Since \(e^{0} = 1\) by Rudin 8.27. And \(e^{\log y} = y\) for all positive \(y\) by Rudin 8.36. Therefore, \(\log(1) = 0\).
 And it's clear that 
 \begin{align*}
     0 =\sum_{k=1}^{\infty} (-1)^{k-1} \frac{0^k}{k}.
 \end{align*}
 Thus, the equation is true in this case.

 Now suppose \(x \in (0,1)\).
 Then for any \(t \in [1, 1+x]\), \(\abs{1-t} = t-1 \leq x < 1\).
 Therefore, by the formula for geometric series,
 \begin{align*}
     \frac{1}{t} = \frac{1}{1-(1-t)} = \sum_{k=0}^{\infty} (1-t)^k.
 \end{align*}
 Therefore,
 \begin{align*}
     \int_{1}^{1+x} \frac{1}{t} \dt = \int_{1}^{1+x} \sum_{k=0}^{\infty} (1-t)^k \dt.
 \end{align*}
 Furthermore, note that the above infinite series has radius of convergence \((0,2)\)  as \(\abs{1-t} < 1 \implies 0<t<2\).
 Thus that by Rudin Theorem 8.1, \(\sum_{k=0}^{\infty} (1-t)^k\) converges uniformly  on \([1, 1 + x]\), as \([1, 1 + x] \subsetneq (0,2)\).
 Hence, by Rudin Theorem 7.16,
 \begin{align*}
     \int_{1}^{1+x} \sum_{k=0}^{\infty} (1-t)^k \dt = \sum_{k=0}^{\infty} \int_{1}^{1+x} (1-t)^k \dt. 
 \end{align*}
 Combining the above results, and evaluating each integrals, we have
 \begin{align*}
     \int_{1}^{1+x} \frac{1}{t} \dt &= \sum_{k=0}^{\infty} \int_{1}^{1+x} (1-t)^k \dt \\
     &= \sum_{k=0}^{\infty} \brac{\frac{-(1-t)^{k+1}}{k+1}}_{1}^{1+x} \\
     &= \sum_{k=0}^{\infty} \frac{-(-x)^{k+1}}{k+1}\\
     &=\sum_{ k=1}^{\infty} \frac{-(-x)^{k}}{k}\\
     &= \sum_{k=1}^{\infty} (-1)^{k-1}\frac{x^k}{k}.
 \end{align*}
 And since Rudin 8.39, \(\log(1+x) = \int_{1}^{1+x} \frac{1}{t} \dt\), the equation is  thus true in this case.

 Now suppose \(x \in (-1,0)\).
 Then again for any \(t \in [1+x, 1]\), \(\abs{1-t} = 1-t \leq -x < 1\).
 And the series \(\sum_{k=0}^{\infty} (1-t)^k\) converges uniformly on \([1+x,1]\), as  \([1, 1+x] \subsetneq (0,2)\).
 Therefore, the above decomposition applies still.
 This completes the proof.
\end{fproof}

\begin{fproof}[1(b)]
 We apply Rudin 8.2 (Abel's Theorem) here.
 First note that \(\sum_{n=1}^{\infty} (-1)^{n-1}1/n\) converges by the alternating series test.
 Therefore, by Rudin 8.2 and by results from (a),
 \begin{align*}
    \lim_{x \to 1^-} \log(1+x) = \lim_{x \to 1^-} \sum_{n=1}^{\infty} (-1)^{n-1} \frac{x^n}{n}=  \sum_{n=1}^{\infty} (-1)^{n-1} \frac{1}{n}.
 \end{align*}
 And since \(e^x\) is and strictly increasing and continuous, then \(\log(y)\), being the inverse of \(e^x\), is also strictly increasing continuous. Therefore,
 \begin{align*}
    \lim_{x \to 1^-} \log(1+x) = \log(2).
 \end{align*}
 This gives that
 \begin{align*}
    \log(2) = \sum_{n=1}^{\infty} (-1)^{n-1} \frac{1}{n}. 
 \end{align*}
\end{fproof}
\newpage

% Problem 2
\begin{fproof}[2]
 First note that since \(\sum_{n=0}^{\infty}a_n = \infty\), then given any \(c > 0\), we also have
 \begin{align*}
    c \cdot \sum_{n=0}^{\infty}a_n = \sum_{n=0}^{\infty}c \cdot a_n = \infty.
 \end{align*}
 Let \(M > 0\) be arbitrary.
 Choose \(c = 1/2\). 
 This means that there exists a natural \(K\) s.t. for \(N > K\),
 \begin{align*}
    \sum_{n=0}^{N} c \cdot a_n > M.
 \end{align*}
 Note that there exists  \(0 < \delta < 1\) such that \((1-\delta)^N = c\), since \(1 - \delta = c^{1/N}\) where \(0 < c^{1/N} < 1\).
 We thus pick this \(\delta\).
 Then note that for any \(1-\delta<x < 1\), and any \(n \leq N\)
 \begin{align*}
    x^n > (1 - \delta)^n = c^{n/N} \geq c,
 \end{align*}
 which, with the fact that each \(a_n \geq 0\), gives that,
 \begin{align*}
    \sum_{n=0}^{N} a_n x^n \geq \sum_{n=0}^{N} c \cdot a_n.
 \end{align*}
 Note also that the partial sums of the series \(\sum_{n=0}^{\infty} a_n\) are monotone increasing on \((0,1)\), since \(a_n \geq 0\) for all \(n\).
 Therefore, for \(0 < x < 1\),
 \begin{align*}
    \sum_{n=0}^{\infty} a_n x^n \geq \sum_{n=0}^{N} a_n x^n.
 \end{align*}
 Therefore, combining the above results, we get that for any \(x \in (1-\delta, 1)\),
 \begin{align*}
    \sum_{n=0}^{\infty} a_n x^n \geq \sum_{n=0}^{N} a_n x^n \geq \sum_{n=0}^{N} c \cdot a_n > M.
 \end{align*}
 This completes the proof that \(\sum_{n=0}^{\infty} a_n x^n \to \infty\) as \(x \to 1^-\).

\end{fproof}
\newpage

% Problem 3
\begin{fproof}[3(a)]
 Note that for any fixed \(i\), the sequence of numbers \((a_{i,1}, a_{i,2}, \cdots)\) contains both \(a_{i,i}\) and \(a_{i,i+1}\), i.e, it contains both 1 and \(-1\) with the rest of entries zero.
 This means that for fixed \(i\), \(\sum_{j=1}^{\infty} = a_{i,j} = 0\).
 Therefore,
 \begin{align*}
    \sum_{i=1}^{\infty} \sum_{j=1}^{\infty} a_{i,j} = 0 + 0 + \cdots = 0.
 \end{align*}
 However, for fixed \(j\), the sequence \((a_{1,j}, a_{2,j}, \cdots)\) contains both non-zero terms \(a_{j,j}\) and \(a_{j-1,j}\) (i.e., 1 and \(-1\)) only if \(j \geq 2\); for \(j =1\), the sequence only contains \(a_{j,j}\) (i.e., 1) as its only non-zero term.
 This means that \(\sum_{i=1}^{\infty}a_{i,1} = 1\), whereas \(\sum_{i=1}^{\infty} a_{i,j} = 0\) for \(j \geq 2\).
 Therefore,
 \begin{align*}
    \sum_{j=1}^{\infty} \sum_{i=1}^{\infty} a_{i,j} 
    &= \sum_{j=1}^{1} \sum_{i=1}^{\infty} a_{i,j} + \sum_{j=2}^{\infty} \sum_{i=1}^{\infty} a_{i,j}\\
    &=\sum_{i=1}^{\infty} a_{i,1} + \sum_{j=2}^{\infty} 0\\
    &= 1 + 0\\
    & = 1.
 \end{align*}
\end{fproof}

\begin{fproof}[3(b)]
 
\end{fproof}

\begin{fproof}[3(c)]

\end{fproof}
\newpage

% Problem 4
\begin{fproof}[4]
 We first show that \(f \in \cR[0,R]\).
 By Rudin Theorem 8.1, we know that \(f\) is continuous on \([0,R)\) since \(f\) is a power series with radius of convergence \(R\).
 Also note that \(f\) is bounded on the whole \([0,R]\) since \(f\) is bounded on \([0,R)\) and \(f(R) \in \R\). 
 This means that \(f(x)\) on \([0,R]\) is a bounded function with possibly \(x = R\) being its only discontinuity. 
 Therefore, by Rudin Theorem 6.10, \(f \in \cR[0,R]\), i.e., the expression \(\int_{0}^{R}f(x) \dx\) is well-defined.

 We next show that the series \(\sum_{n=0}^{\infty} a_n \frac{R^{n+1}}{n+1}\) actually converges.
 First note that for each \(n\),
 \begin{align*}
    a_n \frac{R^{n+1}}{n+1} = a_n R^n \cdot \frac{R }{n+1}.
 \end{align*}
 Let \(\alpha_n = a_n R^n\) and \(\beta_n = R/(n+1)\).
 Since \(\sum_{n=0}^{\infty} \alpha_n\) converges, the partial sums \(A_n \) of \(\sum_{n=0 }^{\infty} \alpha_n \) form a bounded sequence.
 Also \(\beta_0 \geq \beta_1 \geq \beta_n \geq \cdots \), with \(\lim_{n \to \infty } \beta_n = 0.\)
 Therefore, by Rudin Theorem 3.42, \(\sum_{n=0 }^{\infty } \alpha_n \beta_n \) converges, i.e., \(\sum_{n=0}^{\infty }a_n R^{n+1}/(n+1)\) converges.

 Now note that by Rudin Theorem 8.1, \(f(x) = \sum_{n=0 }^{\infty } a_n x^n \) converges uniformly on \([0, R - \xi ]\) given any \(R > \xi > 0\).
 Thus, by Rudin Theorem 7.16,
 \begin{align*}
    \int_{0}^{R - \xi} f(x) \dx = \sum_{n=0}^{\infty} \int_{0}^{R - \xi} a_n x^n \dx.
 \end{align*}
 And by integrating each term,
 \begin{align*}
    \int_{0}^{R - \xi}f(x) \dx = \sum_{n=0}^{\infty}a_n \frac{(R-\xi)^{n+1}}{n+1}.
 \end{align*}
 For the rest of this proof, we show that as \(\xi \to 0^+\),
 \begin{align*}
    \int_{0}^{R - \xi}f(x) \dx \to \int_{0}^{R}f(x) \dx, \text{  and  } 
    \sum_{n=0}^{\infty}a_n \frac{(R-\xi)^{n+1}}{n+1} \to \sum_{n=0}^{\infty}a_n \frac{R^{n+1}}{n+1}.
 \end{align*}
 And since the functional limit is unique, this would complete the proof that \(\int_{0}^{R}f(x) \dx = \sum_{n=0}^{\infty}a_n \frac{R^{n+1}}{n+1}.\)

 Let's show the former first.
 Let \(\varepsilon > 0\) be given. Note first that since \(f(x)\) is bounded on \([0,R]\), then there exists \(M \in \R\) such that \(\abs{f(x)} \leq M\) for all \(x \in [0, R]\).
 Let \(\delta = \varepsilon/M\).
 Then for any \(0 < \xi < \delta\),
 \begin{align*}
    \abs{\int_{0}^{R - \xi} f(x) \dx - \int_{0}^{R} f(x) \dx} 
    &= \abs{\int_{R-\xi}^{R}f(x) \dx} \\
    &\leq \int_{R - \xi}^{R} \abs{f(x)} \dx \tag{Rudin 6.13(b)}\\
    &\leq \int_{R - \xi}^{R }M \dx\\
    & = M \xi \\
    & < M \cdot \frac{\varepsilon}{M}\\
    & < \varepsilon.
 \end{align*}
 This completes the proof of the former functional limit. 

 Let's move on to the later.
 Notice that if we do a change of variable with \(t = (R - \xi)/\xi\),
 \begin{align*}
    \sum_{n=0}^{\infty} a_n \frac{(R - \xi)^{n+1}}{n+1} = \sum_{n=0}^{\infty}a_n \frac{R^{n+1}}{n+1} \pare{\frac{R - \xi}{R}}^{n+1} = \sum_{n=0}^{\infty}a_n \frac{R^{n+1}}{n+1} t^{n+1},
 \end{align*}
 where \(0 < t < 1\).
 Note that as \(\sum_{n=0}^{\infty} a_n \frac{R^{n+1}}{n+1}\) converges by above, by Rudin Theorem 8.2,
 \begin{align*}
    \lim_{t \to 1^-} \sum_{n=0}^{\infty} a_n \frac{R^{n+1}}{n+1} t^{n+1} = \sum_{n=0}^{\infty} a_n \frac{R^{n+1}}{n+1}.
 \end{align*}
 Also note that by the change of variable,
 \begin{align*}
    \lim_{t \to 1^-} \sum_{n=0}^{\infty} a_n \frac{R^{n+1}}{n+1} t^{n+1} = \lim_{\xi \to 0^+} \sum_{n=0}^{\infty}a_n \frac{R^{n+1}}{n+1} \pare{\frac{R - \xi}{R}}^{n+1}.
 \end{align*}
 Therefore, by combining the above results,
 \begin{align*}
    \lim_{\xi \to 0^+} \sum_{n=0}^{\infty} a_n \frac{(R - \xi)^{n+1}}{n+1} = \sum_{n=0}^{\infty} a_n \frac{R^{n+1}}{n+1}.
 \end{align*}
 This completes the latter functional limit, which completes the whole proof.
\end{fproof}
\newpage

% Problem 5
\begin{fproof}[5(a)]

\end{fproof}

\begin{fproof}[5(b)]

\end{fproof}

\begin{fproof}[5(c)]

\end{fproof}
\end{document}