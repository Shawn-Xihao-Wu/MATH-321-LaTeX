\documentclass[12pt]{article}
\usepackage{parskip} % no auto indent

% My marco file
\usepackage{mymarco}

% Begin page style
\usepackage[margin=1.0in]{geometry}
\usepackage{fancyhdr}
\pagestyle{fancy}
\setlength{\headheight}{15pt}
\rhead{36123040 Shawn Wu} % Define header
\lhead{MATH 321 HW 07} % Put assignment #
\cfoot{\thepage}

\begin{document}

% Problem 1
\begin{fproof}[1(a)]
 Fix any \(x\) where \(\abs{x} < 1\).
 First if \(x = 0\). Since \(e^{0} = 1\) by Rudin 8.27. And \(e^{\log y} = y\) for all positive \(y\) by Rudin 8.36. Therefore, \(\log(1) = 0\).
 And it's clear that 
 \begin{align*}
     0 =\sum_{k=1}^{\infty} (-1)^{k-1} \frac{0^k}{k}.
 \end{align*}
 Thus, the equation is true in this case.

 Now suppose \(x \in (0,1)\).
 Then for any \(t \in [1, 1+x]\), \(\abs{1-t} = t-1 \leq x < 1\).
 Therefore, by the formula for geometric series,
 \begin{align*}
     \frac{1}{t} = \frac{1}{1-(1-t)} = \sum_{k=0}^{\infty} (1-t)^k.
 \end{align*}
 Therefore,
 \begin{align*}
     \int_{1}^{1+x} \frac{1}{t} \dt = \int_{1}^{1+x} \sum_{k=0}^{\infty} (1-t)^k \dt.
 \end{align*}
 Furthermore, note that the above infinite series has radius of convergence \((0,2)\)  as \(\abs{1-t} < 1 \implies 0<t<2\).
 Thus that by Rudin Theorem 8.1, \(\sum_{k=0}^{\infty} (1-t)^k\) converges uniformly  on \([1, 1 + x]\), as \([1, 1 + x] \subsetneq (0,2)\).
 Hence, by Rudin Theorem 7.16,
 \begin{align*}
     \int_{1}^{1+x} \sum_{k=0}^{\infty} (1-t)^k \dt = \sum_{k=0}^{\infty} \int_{1}^{1+x} (1-t)^k \dt. 
 \end{align*}
 Combining the above results, and evaluating each integrals, we have
 \begin{align*}
     \int_{1}^{1+x} \frac{1}{t} \dt &= \sum_{k=0}^{\infty} \int_{1}^{1+x} (1-t)^k \dt \\
     &= \sum_{k=0}^{\infty} \brac{\frac{-(1-t)^{k+1}}{k+1}}_{1}^{1+x} \\
     &= \sum_{k=0}^{\infty} \frac{-(-x)^{k+1}}{k+1}\\
     &=\sum_{ k=1}^{\infty} \frac{-(-x)^{k}}{k}\\
     &= \sum_{k=1}^{\infty} (-1)^{k-1}\frac{x^k}{k}.
 \end{align*}
 And since Rudin 8.39, \(\log(1+x) = \int_{1}^{1+x} \frac{1}{t} \dt\), the equation is  thus true in this case.

 Now suppose \(x \in (-1,0)\).
 Then again for any \(t \in [1+x, 1]\), \(\abs{1-t} = 1-t \leq -x < 1\).
 And the series \(\sum_{k=0}^{\infty} (1-t)^k\) converges uniformly on \([1+x,1]\), as  \([1, 1+x] \subsetneq (0,2)\).
 Therefore, the above decomposition applies still.
 This completes the proof.
\end{fproof}

\begin{fproof}[1(b)]
 We apply Rudin 8.2 (Abel's Theorem) here.
 First note that \(\sum_{n=1}^{\infty} (-1)^{n-1}1/n\) converges by the alternating series test.
 Therefore, by Rudin 8.2 and by results from (a),
 \begin{align*}
    \lim_{x \to 1^-} \log(1+x) = \lim_{x \to 1^-} \sum_{n=1}^{\infty} (-1)^{n-1} \frac{x^n}{n}=  \sum_{n=1}^{\infty} (-1)^{n-1} \frac{1}{n}.
 \end{align*}
 And since \(e^x\) is and strictly increasing and continuous, then \(\log(y)\), being the inverse of \(e^x\), is also strictly increasing continuous. Therefore,
 \begin{align*}
    \lim_{x \to 1^-} \log(1+x) = \log(2).
 \end{align*}
 This gives that
 \begin{align*}
    \log(2) = \sum_{n=1}^{\infty} (-1)^{n-1} \frac{1}{n}. 
 \end{align*}
\end{fproof}
\newpage

% Problem 2
\begin{fproof}[2]
 First note that since \(\sum_{n=0}^{\infty}a_n = \infty\), then given any \(c > 0\), we also have
 \begin{align*}
    c \cdot \sum_{n=0}^{\infty}a_n = \sum_{n=0}^{\infty}c \cdot a_n = \infty.
 \end{align*}
 Let \(M > 0\) be arbitrary.
 Choose \(c = 1/2\). 
 This means that there exists a natural \(K\) s.t. for \(N > K\),
 \begin{align*}
    \sum_{n=0}^{N} c \cdot a_n > M.
 \end{align*}
 Note that there exists  \(0 < \delta < 1\) such that \((1-\delta)^N = c\), since \(1 - \delta = c^{1/N}\) where \(0 < c^{1/N} < 1\).
 We thus pick this \(\delta\).
 Then note that for any \(1-\delta<x < 1\), and any \(n \leq N\)
 \begin{align*}
    x^n > (1 - \delta)^n = c^{n/N} \geq c,
 \end{align*}
 which, with the fact that each \(a_n \geq 0\), gives that,
 \begin{align*}
    \sum_{n=0}^{N} a_n x^n \geq \sum_{n=0}^{N} c \cdot a_n.
 \end{align*}
 Note also that the partial sums of the series \(\sum_{n=0}^{\infty} a_n\) are monotone increasing on \((0,1)\), since \(a_n \geq 0\) for all \(n\).
 Therefore, for \(0 < x < 1\),
 \begin{align*}
    \sum_{n=0}^{\infty} a_n x^n \geq \sum_{n=0}^{N} a_n x^n.
 \end{align*}
 Therefore, combining the above results, we get that for any \(x \in (1-\delta, 1)\),
 \begin{align*}
    \sum_{n=0}^{\infty} a_n x^n \geq \sum_{n=0}^{N} a_n x^n \geq \sum_{n=0}^{N} c \cdot a_n > M.
 \end{align*}
 This completes the proof that \(\sum_{n=0}^{\infty} a_n x^n \to \infty\) as \(x \to 1^-\).

\end{fproof}
\newpage

% Problem 3
\begin{fproof}[3(a)]
 Note that for any fixed \(i\), the sequence of numbers \((a_{i,1}, a_{i,2}, \cdots)\) contains both \(a_{i,i}\) and \(a_{i,i+1}\), i.e, it contains both 1 and \(-1\) with the rest of entries zero.
 This means that for fixed \(i\), \(\sum_{j=1}^{\infty} = a_{i,j} = 0\).
 Therefore,
 \begin{align*}
    \sum_{i=1}^{\infty} \sum_{j=1}^{\infty} a_{i,j} = 0 + 0 + \cdots = 0.
 \end{align*}
 However, for fixed \(j\), the sequence \((a_{1,j}, a_{2,j}, \cdots)\) contains both non-zero terms \(a_{j,j}\) and \(a_{j-1,j}\) (i.e., 1 and \(-1\)) only if \(j \geq 2\); for \(j =1\), the sequence only contains \(a_{j,j}\) (i.e., 1) as its only non-zero term.
 This means that \(\sum_{i=1}^{\infty}a_{i,1} = 1\), whereas \(\sum_{i=1}^{\infty} a_{i,j} = 0\) for \(j \geq 2\).
 Therefore,
 \begin{align*}
    \sum_{j=1}^{\infty} \sum_{i=1}^{\infty} a_{i,j} 
    &= \sum_{j=1}^{1} \sum_{i=1}^{\infty} a_{i,j} + \sum_{j=2}^{\infty} \sum_{i=1}^{\infty} a_{i,j}\\
    &=\sum_{i=1}^{\infty} a_{i,1} + \sum_{j=2}^{\infty} 0\\
    &= 1 + 0\\
    & = 1.
 \end{align*}
\end{fproof}

\begin{fproof}[3(b)]
 Let \(A = \set{\sum_{i,j \in I} a_{ij} : \text{ finite } I \subseteq \N \times \N}\).
 \textit{Note that in the following we restrict our discussion to the case that \(\sup A < +\infty\); otherwise it's clear that both double sums are infinite as the prompt suggested.}
 We show both that \(\sum_{i=1}^{\infty} \sum_{j=1}^{\infty} a_{ij} = \sup A\) and \(\sum_{j=1}^{\infty} \sum_{i=1}^{\infty} a_{ij} = \sup A\). It's suffice to show the former as the argument for the later is symmetrical.

 Let \(T = \sum_{i=1}^{\infty} \sum_{j=1}^{\infty} a_{ij}\).
 We first claim that if \(T = +\infty\), then \(\sup A = +\infty\) (the proof of the claim right below). The point is that since we've established that \(\sup A < +\infty\), thus \(T < +\infty\).
 Let's prove that claim. 
 Suppose, for the sake of contradiction, \(\sup A = M \in \R\).
 Since \(T = \sum_{i=1}^{\infty} \sum_{j=1}^{\infty} a_{ij} = \infty\), there must exist a natural \(N\) s.t. \(\sum_{i=1}^{N} \sum_{j=1}^{\infty} a_{ij} > 2M\).
 Note that since each \(a_{ij} \geq 0\), then for any fixed \(i\), either \(\sum_{j=1}^{\infty} a_{ij}\) converges or diverges to \(+\infty\) by monotone convergence theorem.

 Now, if there exists \(\sum_{j=1}^{\infty} a_{ij} = \infty\) for some \(i = 1, 2, \cdots, N\), then it's clear that \(\sup A = +\infty\), which is a contradiction.
 If, however, \(\sum_{j=1}^{\infty} a_{ij} < +\infty\) for each \(i = 1, 2, \cdots, N\), then as the tail of convergent series goes to zero, for each \(i\), there exists \(N_i\) s.t. \(\sum_{j=N_i + 1}^{\infty} < M/N\) or \(-\sum_{j=N_i + 1}^{\infty} > -M/N\).
 This means that
 \begin{align*}
    \sum_{i=1}^{N} \sum_{j=1}^{N_i} a_{ij}
    &= \sum_{i=1}^{N} \pare{\sum_{j=1}^{\infty} a_{ij} - \sum_{j=N_i+1}^{\infty} a_{ij}} \\
    &= \sum_{i=1}^{N}\sum_{j=1}^{\infty} a_{ij}- \sum_{i=1}^{N}\sum_{j=N_i+1}^{\infty} a_{ij} \tag{can split cuz all series converges} \\
    &>  2M - N \cdot M/N \\
    &= M,
 \end{align*}
 which contradicts the fact that \(M\) is an upper bound for \(A\).

 Now that \(T \in \R\), we show that
 \begin{enumerate}
    \item \(T\) is an upper bound for the set A,
    \item \(T - \varepsilon\) is not an upper bound for \(A\), for any \(\varepsilon > 0\).
 \end{enumerate}
 The first point is rather clear. 
 If we pick any \(a = \sum_{i,j \in I} a_{ij} \in A\).
 Note that \(a\) only sums over finitely many \(a_{ij}\)'s. 
 And since each \(a_{ij} \geq 0\), \(T\), being the sum of all \(a_{ij}\)'s, is necessarily no smaller than \(a\).
 This gives that \(T \geq a\).
 So \(T\) is indeed an upper bound for \(A\) as \(a\) is arbitrary.

 To show the second point, let \(\varepsilon > 0\) be given.
 Then there must exists a natural \(N\) s.t.
 \begin{align*}
    \sum_{i=1}^{N} \sum_{j=1}^{\infty} a_{ij} > T - \varepsilon/2.
 \end{align*}
 Similarly, for each \(i\), there exists a natural \(N_i\) s.t. 
 \begin{align*}
    \sum_{j=N_i+1}^{\infty} a_{ij} < \varepsilon/(2N) \text{ or } -\sum_{j=N_i+1}^{\infty} a_{ij} > -\varepsilon/(2N).
 \end{align*}
 This means that
 \begin{align*}
    \sum_{i=1}^{N} \sum_{j=1}^{N_i} a_{ij} 
    = \sum_{i=1}^{N} \pare{\sum_{j=1}^{\infty} a_{ij} - \sum_{j=N_i+1}^{N_i} a_{ij}} = \sum_{i=1}^{N}\sum_{j=1}^{\infty} a_{ij} - \sum_{i=1}^{N}\sum_{j=N_i+1}^{N_i} a_{ij} > T - \frac{\varepsilon}{2} -\frac{N \varepsilon}{2N} = T - \varepsilon,
 \end{align*}
 which means that \(T -\varepsilon\) is not an upper bound.
 This completes the proof.
\end{fproof}

\begin{fproof}[3(c)]
 First note that each \(a_{ij}^+, a_{ij}^- \geq 0\).
 Therefore, by using the simple comparison test against \(\sum_{i=1}^{\infty} \sum_{j=1}^{\infty} |a_{ij}|\), we see that both \(\sum_{i=1}^{\infty}\sum_{j=1}^{\infty} a_{ij}^+ \) and \(\sum_{i=1}^{\infty}\sum_{j=1}^{\infty} a_{ij}^-\) must converge.
 And therefore with the results in \(3(b)\),
 \begin{align*}
    \sum_{i=1}^{\infty}\sum_{j=1}^{\infty} a_{ij}^+ = \sum_{j=1}^{\infty}\sum_{i=1}^{\infty} a_{ij}^+  
    \text{ and } 
    \sum_{i=1}^{\infty}\sum_{j=1}^{\infty} a_{ij}^- = \sum_{j=1}^{\infty}\sum_{i=1}^{\infty} a_{ij}^-.
 \end{align*}
 This also means that for every fixed \(i\), \(\sum_{j=1}^{\infty} a_{ij}^+\) converges; for each fixed \(j\), \(\sum_{i=1}^{\infty} a_{ij}^+\) converges, and similarly for \(a_{ij}^-\)'s.

 Therefore, we have
 \begin{align*}
    \sum_{i=1}^{\infty} \sum_{j=1}^{\infty} a_{ij}
    &= \sum_{i=1}^{\infty} \sum_{j=1}^{\infty} (a_{ij}^+ - a_{ij}^-)\\
    & = \sum_{i=1}^{\infty} \pare{\sum_{j=1}^{\infty} a_{ij}^+ - \sum_{j=1}^{\infty}a_{ij}^-} \tag{as each \(\sum_{j=1}^{\infty} a_{ij}^+, \sum_{j=1}^{\infty}a_{ij}^-\) converges}\\
    & = \sum_{i=1}^{\infty}\sum_{j=1}^{\infty}a_{ij}^+ -  \sum_{i=1}^{\infty}\sum_{j=1}^{\infty}a_{ij}^- \tag{as each double sum converges}\\
    & = \sum_{j=1}^{\infty}\sum_{i=1}^{\infty}a_{ij}^+ - \sum_{j=1}^{\infty}\sum_{i=1}^{\infty}a_{ij}^- \tag{by results above}\\
    &= \sum_{j=1}^{\infty} \pare{\sum_{i=1}^{\infty}a_{ij}^+ - \sum_{i=1}^{\infty}a_{ij}^-} \tag{as each \(\sum_{i=1}^{\infty} a_{ij}^+, \sum_{i=1}^{\infty}a_{ij}
    ^-\) converges}\\
    &= \sum_{j=1}^{\infty}\sum_{i=1}^{\infty} (a_{ij}^+ -a_{ij}^- )\\
    &=\sum_{j=1}^{\infty}\sum_{i=1}^{\infty} a_{ij}.
 \end{align*}
 
\end{fproof}
\newpage

% Problem 4
\begin{fproof}[4]
 We first show that \(f \in \cR[0,R]\).
 By Rudin Theorem 8.1, we know that \(f\) is continuous on \([0,R)\) since \(f\) is a power series with radius of convergence \(R\).
 Also note that \(f\) is bounded on the whole \([0,R]\) since \(f\) is bounded on \([0,R)\) and \(f(R) \in \R\). 
 This means that \(f(x)\) on \([0,R]\) is a bounded function with possibly \(x = R\) being its only discontinuity. 
 Therefore, by Rudin Theorem 6.10, \(f \in \cR[0,R]\), i.e., the expression \(\int_{0}^{R}f(x) \dx\) is well-defined.

 We next show that the series \(\sum_{n=0}^{\infty} a_n \frac{R^{n+1}}{n+1}\) actually converges.
 First note that for each \(n\),
 \begin{align*}
    a_n \frac{R^{n+1}}{n+1} = a_n R^n \cdot \frac{R }{n+1}.
 \end{align*}
 Let \(\alpha_n = a_n R^n\) and \(\beta_n = R/(n+1)\).
 Since \(\sum_{n=0}^{\infty} \alpha_n\) converges, the partial sums \(A_n \) of \(\sum_{n=0 }^{\infty} \alpha_n \) form a bounded sequence.
 Also \(\beta_0 \geq \beta_1 \geq \beta_n \geq \cdots \), with \(\lim_{n \to \infty } \beta_n = 0.\)
 Therefore, by Rudin Theorem 3.42, \(\sum_{n=0 }^{\infty } \alpha_n \beta_n \) converges, i.e., \(\sum_{n=0}^{\infty }a_n R^{n+1}/(n+1)\) converges.

 Now note that by Rudin Theorem 8.1, \(f(x) = \sum_{n=0 }^{\infty } a_n x^n \) converges uniformly on \([0, R - \xi ]\) given any \(R > \xi > 0\).
 Thus, by Rudin Theorem 7.16,
 \begin{align*}
    \int_{0}^{R - \xi} f(x) \dx = \sum_{n=0}^{\infty} \int_{0}^{R - \xi} a_n x^n \dx.
 \end{align*}
 And by integrating each term,
 \begin{align*}
    \int_{0}^{R - \xi}f(x) \dx = \sum_{n=0}^{\infty}a_n \frac{(R-\xi)^{n+1}}{n+1}.
 \end{align*}
 For the rest of this proof, we show that as \(\xi \to 0^+\),
 \begin{align*}
    \int_{0}^{R - \xi}f(x) \dx \to \int_{0}^{R}f(x) \dx, \text{  and  } 
    \sum_{n=0}^{\infty}a_n \frac{(R-\xi)^{n+1}}{n+1} \to \sum_{n=0}^{\infty}a_n \frac{R^{n+1}}{n+1}.
 \end{align*}
 And since the functional limit is unique, this would complete the proof that \(\int_{0}^{R}f(x) \dx = \sum_{n=0}^{\infty}a_n \frac{R^{n+1}}{n+1}.\)

 Let's show the former first.
 Let \(\varepsilon > 0\) be given. Note first that since \(f(x)\) is bounded on \([0,R]\), then there exists \(M \in \R\) such that \(\abs{f(x)} \leq M\) for all \(x \in [0, R]\).
 Let \(\delta = \varepsilon/M\).
 Then for any \(0 < \xi < \delta\),
 \begin{align*}
    \abs{\int_{0}^{R - \xi} f(x) \dx - \int_{0}^{R} f(x) \dx} 
    &= \abs{\int_{R-\xi}^{R}f(x) \dx} \\
    &\leq \int_{R - \xi}^{R} \abs{f(x)} \dx \tag{Rudin 6.13(b)}\\
    &\leq \int_{R - \xi}^{R }M \dx\\
    & = M \xi \\
    & < M \cdot \frac{\varepsilon}{M}\\
    & < \varepsilon.
 \end{align*}
 This completes the proof of the former functional limit. 

 Let's move on to the later.
 Notice that if we do a change of variable with \(t = (R - \xi)/\xi\),
 \begin{align*}
    \sum_{n=0}^{\infty} a_n \frac{(R - \xi)^{n+1}}{n+1} = \sum_{n=0}^{\infty}a_n \frac{R^{n+1}}{n+1} \pare{\frac{R - \xi}{R}}^{n+1} = \sum_{n=0}^{\infty}a_n \frac{R^{n+1}}{n+1} t^{n+1},
 \end{align*}
 where \(0 < t < 1\).
 Note that as \(\sum_{n=0}^{\infty} a_n \frac{R^{n+1}}{n+1}\) converges by above, by Rudin Theorem 8.2,
 \begin{align*}
    \lim_{t \to 1^-} \sum_{n=0}^{\infty} a_n \frac{R^{n+1}}{n+1} t^{n+1} = \sum_{n=0}^{\infty} a_n \frac{R^{n+1}}{n+1}.
 \end{align*}
 Now we do the change of variable again,
 \begin{align*}
    \lim_{t \to 1^-} \sum_{n=0}^{\infty} a_n \frac{R^{n+1}}{n+1} t^{n+1} = \lim_{\xi \to 0^+} \sum_{n=0}^{\infty}a_n \frac{R^{n+1}}{n+1} \pare{\frac{R - \xi}{R}}^{n+1}.
 \end{align*}
 Therefore, by combining the above results,
 \begin{align*}
    \lim_{\xi \to 0^+} \sum_{n=0}^{\infty} a_n \frac{(R - \xi)^{n+1}}{n+1} = \sum_{n=0}^{\infty} a_n \frac{R^{n+1}}{n+1}.
 \end{align*}
 This completes the latter functional limit, which completes the whole proof.
\end{fproof}
\newpage

% Problem 5
\begin{fproof}[5(a)]
 Consider 
 \begin{align*}
    f(x) = \sum_{n=1}^{\infty} a_n x^n = \sum_{n=1}^{\infty} (-1)^{n-1}\frac{1}{n} x^n.
 \end{align*}
 Using ratio test, we get,
 \begin{align*}
    \lim_{n \to \infty} \abs{\frac{a_{n+1}}{a_n}} = \lim_{n \to \infty} \frac{n}{n+1} = 1,
 \end{align*}
 which gives that the radius of convergence is \(1\).
 
 At \(x = 1\), \(f(1) = \sum_{n=1}^{\infty} (-1)^{n-1} \frac{1}{n} = \log(2)\) as proven in 1(b).
 So the power series \(f(x)\) converges at \(x = 1\).
 However, 
 \begin{align*}
    \sum_{n=1}^{\infty} n a_n =\sum_{n=1}^{\infty} (-1)^{n-1} \frac{n}{n} = \sum_{n=1}^{\infty} (-1)^{n-1},
 \end{align*}
 which diverges.

\end{fproof}

\begin{fproof}[5(b)]
 Consider
 \begin{align*}
    f(x) = \sum_{n=1}^{\infty} a_n x^n = \sum_{n=1}^{\infty} \frac{1}{n^2} x^n.
 \end{align*}
 Using ratio test, we get,
 \begin{align*}
    \lim_{n \to \infty} \abs{\frac{a_{n+1}}{a_n}} = \lim_{n \to \infty} \pare{\frac{n}{n+1}}^2 = 1^2 = 1,
 \end{align*}
 which gives that the radius of convergence is \(1\).

 Now let \(f_n(x) = x^n/n^2\) defined on \([-1,1]\).
 Let \(M_n = 1/n^2\).
 Then it's clear that for each \(n\), and each \(x\) on \([-1,1]\)
 \begin{align*}
    \abs{f_n(x)} = \frac{\abs{x}^n}{n^2} \leq \frac{1}{n^2} = M_n.
 \end{align*}
 And \(\sum_{n=0}^{\infty} M_n\) converges by the \(p\)-test.
 Therefore, by Rudin Theorem 7.10 (\(M\)-test), \(f(x) = \sum_{n=0}^{\infty}f_n\) converges uniformly on \([-1,1]\).
\end{fproof}

\begin{fproof}[5(c)]
 Consider the geometric series
 \begin{align*}
    f(x) = \sum_{n=0}^{\infty} a_n x^n = \sum_{n=0}^{\infty} x^n,
 \end{align*}
 which we know already has the radius of convergence \(1\).

 Also note that on \((-1,1)\), \(f(x) = \frac{1}{1-x}\), which is an unbounded function (diverges to \(+\infty\)).
 However, for each partial sum of \(f(x)\), \(S_m(x) = \sum_{n=1}^{m} x^n\) is a bounded function on \((-1,1)\).
 Therefore, for any \(m \in \N\), there will be an \(x \in (-1,1)\) s.t. the difference between \(f(x)\) and \(S_m\) would be at least 1.
 This means that the power series doesn't converge uniformly to \(f(x)\) on \((-1,1)\).

%  This means that \(f\) satisfy the Cauchy Criteria, i.e., 
%  for any \(\varepsilon > 0\), there exists a natural \(N\) s.t. \(p,q \geq N\) (assuming \(q > p\)) implies that
%  \begin{align*}
%     \sum_{n=p}^{q} x^n < \varepsilon,
%  \end{align*}
%  for all \(x \in (-1,1)\).
%  However, note that
%  \begin{align*}
%     \sum_{n=p}^{q} x^n = \sum_{n=1}^{q} x^n - \sum_{n=1}^{p-1}x^n =  \frac{x^{p} - x^{q+1}}{1-x} = \frac{x^p(1 - x^{q+1-p})}{1-x}
\end{fproof}
\end{document}