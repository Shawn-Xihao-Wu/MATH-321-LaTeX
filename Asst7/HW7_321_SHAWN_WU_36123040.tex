\documentclass[12pt]{article}
\usepackage{parskip} % no auto indent
\usepackage{graphicx} % graphics

% My marco file
\usepackage{mymarco}

% Begin page style
\usepackage[margin=1.0in]{geometry}
\usepackage{fancyhdr}
\pagestyle{fancy}
\setlength{\headheight}{15pt}
\rhead{36123040 Shawn Wu} % Define header
\lhead{MATH 321 HW 07} % Put assignment #
\cfoot{\thepage}

\begin{document}

% Problem 1
\begin{fproof}[1(a)]
Fix any \(x\) where \(\abs{x} < 1\).
First if \(x = 0\). Since \(e^{0} = 1\) by Rudin 8.27. And \(e^{\log y} = y\) for all positive \(y\) by Rudin 8.36. Therefore, \(\log(1) = 0\).
And it's clear that 
\begin{align*}
    0 =\sum_{k=1}^{\infty} (-1)^{k-1} \frac{0^k}{k}.
\end{align*}
Thus, the equation is true in this case.

Now suppose \(x \in (0,1)\).
Then for any \(t \in [1, 1+x]\), \(\abs{1-t} = t-1 \leq x < 1\).
Therefore, by the formula for geometric series,
\begin{align*}
    \frac{1}{t} = \frac{1}{1-(1-t)} = \sum_{k=0}^{\infty} (1-t)^k.
\end{align*}
Therefore,
\begin{align*}
    \int_{1}^{1+x} \frac{1}{t} \dt = \int_{1}^{1+x} \sum_{k=0}^{\infty} (1-t)^k \dt.
\end{align*}
Furthermore, note that the above infinite series has radius of convergence \((0,2)\) as \(\abs{1-t} < 1 \implies 0<t<2\).
Thus that by Rudin Theorem 8.1, \(\sum_{k=0}^{\infty} (1-t)^k\) converges uniformly on \([1, 1 + x]\), as \([1, 1 + x] \subsetneq (0,2)\).
Hence, by Rudin Theorem 7.16,
\begin{align*}
    \int_{1}^{1+x} \sum_{k=0}^{\infty} (1-t)^k \dt = \sum_{k=0}^{\infty} \int_{1}^{1+x} (1-t)^k \dt. 
\end{align*}
Combining the above results, and evaluating each integrals, we have
\begin{align*}
    \int_{1}^{1+x} \frac{1}{t} \dt &= \sum_{k=0}^{\infty} \int_{1}^{1+x} (1-t)^k \dt \\
    &= \sum_{k=0}^{\infty} \brac{\frac{-(1-t)^{k+1}}{k+1}}_{1}^{1+x} \\
    &= \sum_{k=0}^{\infty} \frac{-(-x)^{k+1}}{k+1}\\
    &=\sum_{k=1}^{\infty} \frac{-(-x)^{k}}{k}\\
    &= \sum_{k=1}^{\infty} (-1)^{k-1}\frac{x^k}{k}.
\end{align*}
And since Rudin 8.39, \(\log(1+x) = \int_{1}^{1+x} \frac{1}{t} \dt\), the equation is thus true in this case.

Now suppose \(x \in (-1,0)\).
Then again for any \(t \in [1+x, 1]\), \(\abs{1-t} = 1-t \leq -x < 1\).
And the series \(\sum_{k=0}^{\infty} (1-t)^k\) converges uniformly on \([1+x,1]\), as \([1, 1+x] \subsetneq (0,2)\).
Therefore, the above decomposition applies still.
This completes the proof.
\end{fproof}

\begin{fproof}[1(b)]

\end{fproof}
\newpage

% Problem 2
\begin{fproof}[2]

\end{fproof}
\newpage

% Problem 3
\begin{fproof}[3(a)]
  
\end{fproof}

\begin{fproof}[3(b)]
  
\end{fproof}

\begin{fproof}[3(c)]
  
\end{fproof}
\newpage

% Problem 4
\begin{fproof}[4]

\end{fproof}
\newpage

% Problem 5
\begin{fproof}[5(a)]

\end{fproof}

\begin{fproof}[5(b)]

\end{fproof}

\begin{fproof}[5(c)]

\end{fproof}
\end{document}