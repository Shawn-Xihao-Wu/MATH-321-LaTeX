\documentclass[12pt]{article}
\usepackage{parskip} % no auto indent
\usepackage{graphicx} % graphics

% My marco file
\usepackage{mymarco}

% Begin page style
\usepackage[margin=1.0in]{geometry}
\usepackage{fancyhdr}
\pagestyle{fancy}
\setlength{\headheight}{15pt}
\rhead{36123040 Shawn Wu} % Define header
\lhead{MATH 321 HW 03} % Put assignment #
\cfoot{\thepage}

\begin{document}

% Problem 1
\begin{fproof}[1(a)]
    Following the convention in Rudin Theorem 6.19 (Change of Variable), we let \(\varphi(t) = e^t\).
    Moreoever,
    \begin{align*}
        f(x) &= \cos(x), &\alpha(x) &= \ln(x), \text{ and }\\
        g(t) &= f(\varphi(t)) = \cos(e^t)
        , &\beta(t) &= \alpha(\varphi(t)) = \ln(e^t) = t.
    \end{align*}
    Therefore, by Theorem 6.19,
    \begin{align*}
        \int_{a}^{b} \cos(e^t) \dt = \int_{a}^{b} g(t)\dbet(t) = \int_{\varphi(a)}^{\varphi(b)} f(x) \dalf(x) = \int_{e^a}^{e^b} f(x) \dalf(x).
    \end{align*}
    Also note that the change of integrand from \(\beta(t)\) to \(\alpha(x)\) makes sense since the domain of the natural log, \(\ln\), is \((0,+\infty)\) and we are assuming \([a,b] \subseteq (0,+\infty)\).

    Now note that \(\alpha'(x) = 1/x \in \mathcal{R}[a,b]\) where \([a,b] \subseteq  (0, +\infty)\).
    Therefore, by Rudin Theorem 6.17,
    \begin{align*}
         \int_{e^a}^{e^b} f(x) \dalf = \int_{e^a}^{e^b} f(x) \alpha'(x) \dx = \int_{e^a}^{e^b} \frac{\cos(x)}{x} \dx.
    \end{align*}

    Now we apply Rudin Theorem 6.22 (Integration by Parts),
    letting \(g(x) = \cos(x), F(x) = 1/x\).
    Hence, \(G(x) = \sin(x), f(x) = -1/x^2\).
    Therefore,
    \begin{align*}
        \int_{e^a}^{e^b} \frac{\cos(x)}{x} \dx = \brac{\frac{\sin(x)}{x}}_{e^a}^{e^b} + \int_{e^a}^{e^b} \frac{\sin(x)}{x^2} \dx = e^{-b}\sin(e^b) - e^{-a}\sin(e^a) + \int_{e^a}^{e^b} \frac{\sin(x)}{x^2} \dx.
    \end{align*}
    Let \(r(a,b) = \int_{e^a}^{e^b} \frac{\sin(x)}{x^2} \dx\).
    Note that
    \begin{align*}
        \abs{r(a,b)}& = \abs{\int_{e^a}^{e^b} \frac{\sin(x)}{x^2} \dx} \\
        & \leq \int_{e^a}^{e^b} \abs{\frac{\sin(x)}{x^2}} \dx \tag{by Rudin Theorem 6.13(b)}\\
        & = \int_{e^a}^{e^b} \frac{\abs{\sin(x)}}{x^2} \dx \tag{as \(x > 0\) on \([e^a, e^b]\)}\\
        &\leq \int_{e^a}^{e^b} \frac{1}{x^2}\dx \tag{as \(\abs{\sin(x)} \leq 1\)}\\
        &=\brac{-x^{-1}}_{e^a}^{e^b}\\
        &=e^{-a} - e^{-b}.
    \end{align*}
    Therefore, we have recovered the equation in question that,
    \begin{align*}
        \int_{a}^{b} \cos(e^t)\dt = e^{-b}\sin(e^{b}) - e^{-a}\sin(e^a) + r(a,b),
    \end{align*}
    where \(\abs{r(a,b)} \leq e^{-a} - e^{-b}\).
\end{fproof}

\begin{fproof}[1(b)]
    Let \(\varepsilon>0\) be given.
    Note that for any natural numbers \(n,m\),
    \begin{align*}
        \abs{I_n - I_m} &= \abs{\int_{m}^{n} \cos(e^t) \dt} = \abs{e^{-n}\sin(e^n) - e^{-m}\sin(e^m) + r(m,n)}\\
        & \leq \abs{e^{-n}\sin(e^n)} + \abs{e^{-m}\sin(e^m)} + \abs{r(m,n)}\\
        & \leq e^{-n} \cdot 1 + e^{-m} \cdot 1 + e^{-m} - e^{-n}\\
        & =2e^{-m}.
    \end{align*}
    Therefore, choose a natural \(N\) such that \(e^{N} > 2/\varepsilon\).
    Then for all \(n,m \geq N\),
    \(e^{m} \geq e^{N} > 2/\varepsilon \implies e^{-m} < \varepsilon/2 \implies 2e^{-m} < \varepsilon \).
    This means that \(\abs{I_n - I_m} < \varepsilon\).
    \((I_n)_{n \in \mathbb{N}}\) is a Cauchy sequence.
\end{fproof}

\begin{fproof}[1(c)]
Let \(\varepsilon > 0\) be arbitrary.
We need a natural \(N\) such that for any real \(x > N\),
\begin{align*}
    \abs{\int_{0}^{x} \cos(e^t) \dt - L} < \varepsilon.
\end{align*}
Since the sequence \(\int_{0}^{n} \cos(e^t) \dt \to L\), there exists a natural \(N_1\) such that for all \(n > N_1\), \(\abs{\int_{0}^{n} \cos(e^t) \dt - L} < \varepsilon/2.\)
And we choose another \(N_2\) where \(e^{N_2} > 4/\varepsilon\) so that \(n > N_2 \implies e^n > e^{N_2} > 4/\varepsilon \implies 2e^{-n} < \varepsilon/2\).
We thus let \(N = \max\{N_1, N_2\}\).

Note that for any \(x > N\), there exists a natural \(n > N\) where \(x \in [n, n+1]\).
Thus,
\begin{align*}
    \abs{\int_{0}^{x} \cos(e^t) \dt - L} &= \abs{\int_{0}^{n} \cos(e^t) \dt - L + \int_{n}^{x} \cos(e^t) \dt}\\
    &\leq \abs{\int_{0}^{n} \cos(e^t) \dt - L} + \abs{\int_{n}^{x} \cos(e^t) \dt}\\
    & < \frac{\varepsilon}{2} + e^{-x}\sin(e^x) - e^{-n}\sin(e^n) + 
\end{align*}
\end{fproof}
\newpage

% Problem 2
\begin{fproof}[2(a)]

\end{fproof}

\begin{fproof}[2(b)]

\end{fproof}

\begin{fproof}[2(c)]

\end{fproof}

\begin{fproof}[2(d)]

\end{fproof}

\begin{fproof}[2(e)]

\end{fproof}
\end{document}