\documentclass[12pt]{article}

% Required packages
\usepackage[margin=1in]{geometry}
\usepackage{fancyhdr}
\usepackage{titlesec}
\usepackage{amsthm}
\usepackage{amsmath}
\usepackage{amssymb}
\usepackage{bm}
\usepackage{parskip}
\usepackage{graphicx}

% \geometry{left=0.5in, right=0.5in}

% Page style
\pagestyle{fancy}

% Define header
\rhead{36123040 Shawn Wu}
\lhead{MATH 321 HW 01} % Put your actual assignment number here
\cfoot{\thepage}

% To start each section (problem) on a new page
% \titleformat{\section}[block]{\normalfont\bfseries}{\thesection}{1em}{}
% \titlespacing*{\section}{0pt}{\baselineskip}{\baselineskip}
% \newcommand{\sectionbreak}{\clearpage}


\newenvironment{fproof}[1][]
  {\begin{proof}[\ifx\relax#1\relax\else\textbf{\large #1} Proof\fi]}
  {\end{proof}}

% Redefine the QED symbol
\renewcommand{\qedsymbol}{\rule{0.7em}{0.7em}}  % Solid square

\newcommand{\set}[1]{\left\{ #1 \right\}}
\newcommand{\abs}[1]{\left\lvert #1 \right\rvert}
\newcommand{\norm}[1]{\left\lVert #1 \right\rVert}
\newcommand{\angl}[1]{\left\langle #1 \right\rangle}
\newcommand{\pare}[1]{\left( #1 \right)}
\newcommand{\brac}[1]{\left[ #1 \right]}




\begin{document}

% Problem 1
\begin{fproof}[1]
  First notice that since \(f\) is a constant function always equals to 1, \(m_i = M_i = 1\) for all \(i\), for any partition \(P\).
  Therefore, for any partition \(P\),
  \begin{align*}
    U(P,f, \alpha) = \sum_i M_i \Delta \alpha_i = \sum \Delta \alpha_i = \alpha(b) - \alpha(a),
  \end{align*}
  as the sum telescopes.
  Similarly, for any partition \(P\),
  \begin{align*}
    L(P,f, \alpha) = \sum_i m_i \Delta \alpha_i = \sum \Delta \alpha_i = \alpha(b) - \alpha(a).
  \end{align*}
  Since the upper and lower Stieltjes sums both remain the same no matter the partition we choose, hence based on the definition of upper and lower Stieltjes integrals,
  \begin{align*}
    \overline{\int_{a}^{b}} f d \alpha = \alpha(b) - \alpha(a) = \underline{\int_{a}^{b}} f d \alpha.
  \end{align*} 
  And since the upper and lower Stieltjes integrals agree, 
  \begin{align*}
    \int_{a}^{b} f d \alpha = \alpha(b) - \alpha(a).
  \end{align*}
\end{fproof}

\newpage

% Problem 2
\begin{fproof}[2]
\end{fproof}

\newpage

% Problem 3
\begin{fproof}[3]
\end{fproof}

\newpage

% Problem 4
\begin{fproof}[4(a)]
\end{fproof}

\begin{fproof}[4(b)]
\end{fproof}

\begin{fproof}[4(c)]
\end{fproof}

\begin{fproof}[4(d)]
\end{fproof}

\newpage

% Problem 5
\begin{fproof}[5]
\end{fproof}


\end{document}
