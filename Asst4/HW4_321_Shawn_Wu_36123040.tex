\documentclass[12pt]{article}
\usepackage{parskip} % no auto indent
\usepackage{graphicx} % graphics

% My marco file
\usepackage{mymarco}

% Begin page style
\usepackage[margin=1.0in]{geometry}
\usepackage{fancyhdr}
\pagestyle{fancy}
\setlength{\headheight}{15pt}
\rhead{36123040 Shawn Wu} % Define header
\lhead{MATH 321 HW 04} % Put assignment #
\cfoot{\thepage}

\begin{document}

% Problem 1
\begin{fproof}[1(a)]
    We claim that \(\lim_{n\to\infty} f_n(x) = 0\) for all \(x \in \R\).

    First note that if \(x = 0\), then \(f_n = n \cdot 0/(1+n^2 \cdot 0) = 0\) for all \(n\).
    Therefore, \(\lim_{n\to\infty}f_n(0) = 0\).
    Now if \(x \in \R\setminus\{0\}\).
    Then \(nx \neq 0\) for any \(n\), so
    \begin{align*}
        \abs{f_n(x)} 
        = \abs{\frac{nx}{1+n^2x^2}} 
        = \frac{\abs{nx}}{1+|nx|^2}
        = \frac{1}{\frac{1}{|nx|} + |nx|}.
    \end{align*}
    Note that \(1/|nx| + |nx| \to \infty\) as \(n \to \infty\).
    Therefore, \(|f_n(x)| \to 0\) as \(n \to \infty\).
    This proves that \(\set{f_n(x)}_{n \in \N}\) converge pointwisely and the limit is 0.
\end{fproof}

\begin{fproof}[1(b)]
    Before we prove the whole statement, we prove a few useful results that would be helpful.

    We first claim that \(\varphi(x) = |f_N(x)|\) is decreasing on \([1/N, +\infty)\) and increasing on \((-\infty, -1/N]\), for \(N\) fixed.
    Pick any \(p,q \in \R\) such that \(N < p < q\), which also means that \(1/N > 1/p > 1/q\).
    We want to show that \(\varphi(1/p) < \varphi(1/q)\).
    Note that
    \begin{align*}
        \varphi(1/p) &= \frac{N/p}{1 + (N/p)^2} = \frac{1}{\frac{1}{N/p} + N/p} = \frac{1}{\frac{p^2+N^2}{Np}} = \frac{1}{\frac{qp^2+qN^2}{Npq}},\\
        \varphi(1/q) &=\frac{N/q}{1 + (N/q)^2} = \frac{1}{\frac{1}{N/q} + N/q} = \frac{1}{\frac{q^2 + N^2}{Nq}} = \frac{1}{\frac{pq^2 + pN^2}{Nqp}}.
    \end{align*}
    Also note that
    \begin{align*}
        \frac{qp^2 + qN^2}{Npq} - \frac{pq^2 + pN^2}{Npq} = \frac{pq(p-q) + N^2(q-p)}{Npq} = \frac{(q-p)(pq - N^2)}{Npq} > 0,
    \end{align*}
    as \(p < q\), and \(N^2 < pq\) from \(N<p,q\).
    Therefore, 
    \begin{align*}
        \frac{qp^2 + qN^2}{Npq} > \frac{pq^2 + pN^2}{Npq} \implies \frac{1}{\frac{qp^2 + qN^2}{Npq}} < \frac{1}{\frac{pq^2 + pN^2}{Npq}} \implies \varphi(1/p) < \varphi(1/q).
    \end{align*}
    \(\varphi\) is decreasing on \([1/N, +\infty)\) as desired, and by a similar argument, \(\varphi\) is increasing on \((-\infty, -1/N]\).

    Also it's clear that \(\varphi(x)\) is an even function.

    % Also notice that for a fixed \(X \in \R\), \(\psi(n)= |f_n(X)|\) is also a decreasing function on \(\left[\frac{1}{|X|}, +\infty\right)\), since \(|f_n(x)|\) over \(\R_{\geq 0}\) is the same as a function of \(n\) for a fixed \(x\) and as a function of \(x\) for a fixed \(n\).

    \begin{center}
        \(\ast~\ast~\ast\)
    \end{center}

    We now first prove the backward direction.
    Suppose that \(0 \notin A'\).
    Then there exists \(\delta>0\) such that \((-\delta, +\delta) \cap A = \varnothing.\)
    Thus, we can show that \(\set{f_n(x)}\) converges uniformly to 0 on \(A\) by showing that \(\set{f_n(x)}\) uniformly converges to \(0\) on \(\R \setminus (-\delta, +\delta),\) since \(A \subseteq \R \setminus (-\delta, +\delta)\).
    Let \(I = \R \setminus (-\delta, +\delta)\).

    Given \(\varepsilon>0\).
    We want a natural \(N\) such that \(n \geq N\) implies \(|f_n(x)| < \varepsilon\) for all \(x \in I\).
    Consider the points \(x = \pm\delta\).
    First there must be a natural \(N_1\) such that \(\delta \in [1/N_1, +\infty)\).
    Therefore, by previous results \(\varphi(x) = |f_{N_1}(x)|\) is decreasing on \([1/N_1, +\infty)\) and increasing on \((-\infty, -1/N_1]\).
    Since \(-\delta <-1/N_1 < 1/N_1 < \delta\), \(\varphi(x)\) is decreasing on \([\delta, +\infty)\) and increasing on \((-\infty, -\delta]\) as well, i.e., \(\sup_{x \in I} \varphi(x) = \varphi(\delta) = \varphi(-\delta).\)
    And since \(\set{f_n(\delta)} \to 0\) as \(n \to \infty\), there must be another \(N_2\) such that \(n \geq N_2\) implies that \(|f_n(\delta)|=|f_n(-\delta)| < \varepsilon \). 
    We thus take \(N > \max\{N_1, N_2\}\).
    Note that \(N > N_1 \implies 1/N < 1/N_1 < \delta\).
    Hence for any \(n \geq N\), \(\sup_{x\in I} f_n(x) = |f_n(\delta)| = |f_n(-\delta)| < \varepsilon\).
    This proves that the function converges uniformly to zero.

    \begin{center}
        \(\ast~\ast~\ast\)
    \end{center}
    
    We now proceed to the forward direction.
    We do proof by contraposition.
    Suppose that \(0 \in A'\).
    Then there exists a sequence \((a_i)_{i \in \N} \subseteq A\) converges to 0 where \(|a_{i+1}| \leq |a_i|\) for all \(i\).
    To see this, note that there exists \(x_1 \in (-1, 1) \cap A\); set \(a_1 = x_1\), and for any \(i \geq 2\), there exists \(x_i \in (\max\{-1/i, -|x_{i-1}|\}, \min\{1/i, |x_{i-1}|\})\); set \(a_i = x_i\).

    We want to show that \(\set{f_n(x)}\) does not converge uniformly to \(0\), i.e., there exists \(\varepsilon > 0\), and there exist infinitely many \(N\)'s such that for each \(N\), \(|f_N(x)| > \varepsilon\) for some \(x\).

    We pick \(\varepsilon = 1/4\). Now consider the \((a_i)_{i \in \N}\) generated before. Note that for each \(a_i\), there exists a natural \(N\) such that \(|a_i| \in [1/N, 1/(N-1)]\).
    And note that \(\varphi(x)= |f_N(x)|\) is decreasing on \([1/N, +\infty)\) and increasing on \((-\infty, -1/N]\), and note that fact that \(\varphi(x)\) is even.
    Thus,
    \begin{align*}
        f_N\left(\frac{1}{N-1}\right) \leq f_N(a_i) \leq f_N\left(\frac{1}{N}\right).
    \end{align*}
    Also note that for any natural \(N\geq 2\),
    \begin{align*}
        f_N \pare{\frac{1}{N-1}} &= \frac{\frac{N}{N-1}}{1 + \pare{\frac{N}{N-1}}^2} = \frac{1}{\frac{N-1}{N} + \frac{N}{N-1}} = \frac{N(N-1)}{(N-1)^2 + N^2} = \frac{N^2 - N}{2(N^2 - N) + 1} = \frac{1}{2 + \frac{1}{N^2-N}} \\
        &> \varepsilon.
    \end{align*}
    Putting everything together, we know that since \((a_i)\) is such a sequence that its absolute value is decreasing and it converges to zero, then it generates an infinite sequence of \((N_i)_{i\in \N}\) where each \(|a_i| \in [1/N_i, 1/(N_i-1)]\), and there are infinitely many distinct elements in the sequence \((N_i)_{i \in \N}\).
    And for each \(N_i\), \[f_{N_i}(a_i) \geq f_{N_i}\left(\frac{1}{N-1}\right) > \varepsilon.\]
    This completes the proof that \(\set{f_n(x)}\) does not converge uniformly on \(A\).

\end{fproof}
\newpage

% Problem 2
\begin{fproof}[2(a)]
    First note that if \(x = 0\) or \(x = 1\), then for any \(n \in \N\) and any \(x \in [0,1]\), \(f_n(x) = 0\).
    If however, \(x \in (0,1)\), then \(0< 1 - x^4 < 1\), i.e., \(1-x^4 = 1/(1+p)\) for some \(p > 0\).
    Therefore, by Rudin Theorem 3.20(d), for any fixed \(x \in [0,1]\)
    \begin{align*}
        \lim_{n\to\infty} n^c(1-x^4)^n = \lim_{n\to\infty} \frac{n^c}{(1+p)^n} = 0,
    \end{align*}
    so,
    \begin{align*}
        \lim_{n\to\infty}n^cx^3(1-x^4)^n = \lim_{n\to\infty} x^3 \cdot \lim_{n\to\infty} \frac{n^c}{(1+p)^n} = x^3 \cdot 0 = 0.
    \end{align*}
    The limit function \(f(x)\) thus exists and \(f(x) = 0\) for all \(x \in [0,1].\)
\end{fproof}

\begin{fproof}[2(b)]
    We claim that the convergence is not uniform for \(c \geq 3/4\) but it is uniform for \(c < 3/4\).

    For \(c \geq 3/4\), let \(x_n = n^{-1/4}\), then
    \begin{align*}
        f_n(x_n) = n^{c-\frac{3}{4}}\left(1 - \frac{1}{n}\right)^n.
    \end{align*}
    Note that \((1 - 1/n)^n \to e^{-1}\) as \(n \to \infty\).
    Therefore, when \(c = 3/4\), \(f_n(x_n) \to e^{-1}\) as \(n \to \infty\).
    And when \(c > 3/4\), \(n^{c-3/4}\) diverges, so \(f_n(x_n)\) diverges as \(n \to\infty\).
    In either case, \(f_n(x_n)\) is at least \(e^{-1}/2\) for infinitely many \(n\)'s. This concludes that \(\set{f_n(x)}\) cannot be converging to zero uniformly in this case.

    For \(c < 3/4\). First note that \(1+t \leq e^t\) for all \(x \in \R\).
    We also set \(y = nx^4\). Thus,
    \begin{align*}
        \abs{f_n(x)} = \abs{n^cx^3(1-x^4)^n} \leq \abs{n^cx^3e^{-nx^4}} = \abs{n^{c-\frac{3}{4}}(nx^4)^{\frac{3}{4}}e^{-nx^4}} = \abs{n^{c-\frac{3}{4}}y^{\frac{3}{4}}e^{-y}}.
    \end{align*}
    Note that \(y^{3/4}e^{-y} \to 0\) as exponential grows faster than polynomial. Therefore the function \(\varphi(y) =  y^{3/4}e^{-y}\) is bounded.
    Also note that \(n^{c-3/4} \to 0\) as \(n \to \infty\) since \(c-3/4 < 0\).
    Hence, \(\abs{n^{c-3/4}y^{3/4}e^{-y}} \to 0\) as \(n \to \infty\).
    This means that \(\sup_{x\in [0,1]}\abs{f_n(x)} \to 0\), which shows that \(\set{f_n(x)}\) converges uniformly to zero in this case.
\end{fproof}

\begin{fproof}[2(c)]
    First note that \[\int_{0}^{1}f(x) \dx = \int_{0}^{1} 0 \dx = 0.\]
    We then use change of variable and let \(g(x) = 1-x^4\), so \(g'(x) = -4x^3\); note that the derivative is continuous.
    Thus,
    \begin{align*}
        \int_{0}^{1}x^3(1-x^4)^n \dx &= \frac{-1}{4}\int_{0}^{1}(1-x^4)^n(-4x^3) \dx = \frac{-1}{4} \int_{0}^{1} (g(x))^n g'(x) \dx \\
        &= \frac{-1}{4} \int_{g(0)}^{g(1)} u^n \du = \frac{-1}{4}\int_{1}^{0} u^n \du\\
        & = \frac{-1}{4} \brac{\frac{u^{n+1}}{n+1}}_{u=1}^{u=0}\\
        &= \frac{1}{4} \cdot \frac{1}{n+1}.
    \end{align*}
    Therefore, 
    \begin{align*}
        \int_{0}^{1} f_n(x) \dx = n^c \int_{0}^{1}x^3(1-x^4)^n \dx = \frac{n^c}{4(n+1)}.
    \end{align*}
    
    And it's clear that the above integral converges to \(0\) iff \(c < 1\).
\end{fproof}
\newpage

% Problem 3
\begin{fproof}[3]
  Given \(\varepsilon>0\).
  Since \(g\) is continuous at 0, there exists \(\delta>0\) s.t. \(\abs{g(x)- g(0)}<\varepsilon\) whenever \(|x| < \delta\).
  And by (iii), there exists a natural \(N\) s.t. \(f_n(x) < \varepsilon\) for all \(n \geq N\) and for all \(x \in [-1, -\delta] \cup [+\delta, +1]\).
  Also let \(M = \sup_{x \in [-1,1]} g(x)\).
  Therefore, for all \(n \geq N\),
  \begin{align*}
    &\abs{\int_{-1}^{1}f_n(x)g(x) - g(0) \dx}\\
   =& \abs{\int_{-1}^{1}f_n(x)[g(x)-g(0)] \dx}\\
   \leq & \abs{\int_{-1}^{-\delta} f_n(x)[g(x)-g(0)] \dx} + \abs{\int_{-\delta}^{+\delta} f_n(x)[g(x)-g(0)]\dx} + \abs{\int_{+\delta}^{+1} f_n(x)[g(x)-g(0)]\dx}\\
   \leq &\int_{-1}^{-\delta} f_n(x)\abs{g(x)-g(0)}\dx + \int_{-\delta}^{+\delta} f_n(x)\abs{g(x)-g(0)}\dx + \int_{+\delta}^{+1} f_n(x)\abs{g(x)-g(0)}\dx\\
   \leq& \int_{-1}^{-\delta} \varepsilon\abs{g(x)-g(0)}\dx + \int_{-\delta}^{+\delta} f_n(x) \varepsilon \dx + \int_{+\delta}^{+1} \varepsilon\abs{g(x)-g(0)}\dx \tag{by above assumptions}\\
   \leq& \varepsilon\int_{-1}^{-\delta} \abs{g(x)-g(0)}\dx + \varepsilon\int_{-\delta}^{+\delta} f_n(x) \dx + \varepsilon\int_{+\delta}^{+1}\abs{g(x)-g(0)}\dx\\
   \leq & \varepsilon 2M(1-\delta) + \varepsilon \int_{-1}^{1} f_n(x) + \varepsilon 2M(1-\delta) \tag{since \(f_n(x) \geq 0\)}\\
   \leq & \varepsilon4M(1-\delta) + \varepsilon\\
   \leq & \varepsilon4M + \varepsilon\\
   \leq & \varepsilon(4M+1).
  \end{align*}
  This concludes the proof.

\end{fproof}
\newpage

% Problem 4
\begin{fproof}[4(a)]
    First note that for all \(x \in \R\), \(0 \leq \varphi(x) \leq 1\).
    Also note that \(\sum_{n=1}^{\infty}\frac{1}{2^n}\) is a geometric series hence converge and equals \(1\).
    Now fix \(t \in \R\).
    Note that \(f_1(t)\) is a series of non-negative terms that's bounded above, thus
    \begin{align*}
        0 \leq f_1(t) =  \sum_{n=1}^{\infty} \frac{\varphi(3^{2n-2}t)}{2^n} \leq \sum_{n=1}^{\infty} \frac{1}{2^n} = 1.
    \end{align*}
    So, \(f_1(t)\) converges, which means that \(f_1\) is well-defined.
    With a similar reasoning, \(f_2\) is well-defined as well.
    And \(0 \leq f_i(t) \leq 1\) for all \(t\).

    \begin{center}
        \(\ast~\ast~\ast\)
    \end{center}

    We show that \(f_1(t)\) is continuous by showing that it is the limit of a sequence of uniform convergent continuous functions; with a similar reasoning, \(f_2(t)\) is continuous as well.
    Let 
    \begin{align*}
        S_I(t) = \sum_{i=1}^{I} \frac{\varphi(3^{2i-2}t)}{2^i}.
    \end{align*}
    Note that each \(S_I(t)\) is a continuous function on \(\R\), since it is a finite sum of compositions of continuous functions.

    Now let \(\varepsilon > 0\) be given.
    Note that \(\sum_{i=1}^{\infty} 1/2^i\) converges, so there exists a natural \(N\) s.t. \(I, J \geq N\) implies \[\abs{\sum_{i=I}^{J} \frac{1}{2^i}} < \varepsilon.\]
    And note that for any \(I,J \geq N\),
    \begin{align*}
        \abs{S_I(t) - S_J(t)}= \abs{\sum_{i=I}^{J} \frac{\varphi(3^{2i-2})}{2^i} } \leq \abs{\sum_{i=I}^{J} \frac{1}{2^i}} < \varepsilon.
    \end{align*}
    Therefore, by the Cauchy Criterion, \(\set{S_I}_{I\in \N}\) converges uniformly.
    Since \(f_1(t)\) is its limiting function, we conclude that \(f_1(t)\) must be continuous as well, by Rudin Theorem 7.12.

\end{fproof}

\begin{fproof}[4(b)]
    sorry I didn't have the time to finish this question...
\end{fproof}

\begin{fproof}[4(c)]
    We just need to follow the results established before.
    Note that
    \begin{align*}
        f(c) &= (f_1(c), f_2(c))\\
        &= \pare{\sum_{n=1}^{\infty} \frac{\varphi(3^{2n-2}c)}{2^n},\sum_{n=1}^{\infty} \frac{\varphi(3^{2n-1}c)}{2^n}}\\
        &=\pare{\sum_{n=1}^{\infty} \frac{c_{2n-1}}{2^n} ,\sum_{n=1}^{\infty} \frac{c_{2n}}{2^n}} \tag{by 4(b)}\\
        &=\pare{\sum_{n=1}^{\infty} \frac{a_n}{2^n}, \sum_{n=1}^{\infty} \frac{b_n}{2^n}} \tag{by definition of \(c_i\)}\\
        &= (a,b).
    \end{align*}
\end{fproof}
\end{document}