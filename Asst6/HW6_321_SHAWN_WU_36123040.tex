\documentclass[12pt]{article}
\usepackage{parskip} % no auto indent
\usepackage{graphicx} % graphics

% My marco file
\usepackage{mymarco}

% Begin page style
\usepackage[margin=1.0in]{geometry}
\usepackage{fancyhdr}
\pagestyle{fancy}
\setlength{\headheight}{15pt}
\rhead{36123040 Shawn Wu} % Define header
\lhead{MATH 321 HW 06} % Put assignment #
\cfoot{\thepage}

\begin{document}

% Problem 1
\begin{fproof}[1(a)]
    Let \(\Gamma = \set{\cC(X):\norm{f}\leq 1 \tand N_{\alpha}(f) \leq 1}\).
    We appeal to HW5 Problem 3.
    We show that \(\Gamma\) is compact by showing that it is closed, bounded, and equicontinous, as \(\Gamma \subseteq \cC(X)\) and \(X\) is compact.

    \textbf{Closed:}
    We show that \(\Gamma' \subseteq \Gamma\).
    Pick any \(g \in \Gamma'\).
    There exists a sequence of functions \((g_n) \subseteq \Gamma \setminus \{g\}\) that converges to \(g\) w.r.t. the supremum norm, and this means that \(g_n \to g\) uniformly by Rudin Theorem 7.7.

    We show that \(g \in \Gamma\), i.e., \(\norm{g} \leq 1\) and \(N_{\alpha}(g) \leq 1\).

    Given any \(\varepsilon > 0\).
    We know there exists a natural \(N\) s.t. \(\norm{g - g_N} < \varepsilon\).
    Also,
    \begin{align*}
        \norm{g} = \norm{g - g_N + g_N} \leq \norm{g-g_N} + \norm{g_N} < 1 + \varepsilon.
    \end{align*}
    Since \(\varepsilon\) is arbitrary, \(\norm{g} \leq 1\).

    Now again given any \(\varepsilon > 0\).
    Pick any \(x,y \in X\) where \(x \neq y\).
    Then first note that \(d(x,y)>0\) since \(x\neq y\), so \(d(x,y)^{\alpha}\) is a real number greater than zero.
    And since \(g_n \to g\) uniformly, there exists a natural \(N\) s.t., \(\abs{g_N(t)-g(t)} < \varepsilon \cdot d(x,y)^{\alpha}/2\), for all \(t \in X\).
    Therefore,
    \begin{align*}
        \frac{\abs{g(x)-g(y)}}{d(x,y)^{\alpha}} &= \frac{\abs{g(x)-g_N(x) + g_N(y) -g(y) + g_N(x)-g_N(y)}}{d(x,y)^{\alpha}}\\
        & \leq \frac{\abs{g(x)-g_N(x)}}{d(x,y)^{\alpha}} + \frac{\abs{g_N(y)-g(y)}}{d(x,y)^{\alpha}} + \frac{\abs{g_N(x)-g_N(y)}}{d(x,y)^{\alpha}}\\
        &<\frac{\varepsilon \cdot d(x,y)^{\alpha}}{2 d(x,y)^{\alpha}} + \frac{\varepsilon \cdot d(x,y)^{\alpha}}{2 d(x,y)^{\alpha}} + 1\\
        &= 1 + \varepsilon.
    \end{align*}
    Since \(\varepsilon\) is arbitrary, \(\abs{g(x)-g(y)}/d(x,y)^{\alpha} \leq 1\) for this particular pair of \(x\) and \(y\).
    And since \(x,y\) is arbitrary,
    then 1 is a upper bound of the set \(A\) where 
    \begin{align*}
        A = \set{\frac{\abs{g(x)-g(y)}}{d(x,y)^{\alpha}} : x,y \in X, x\neq y},
    \end{align*}
    which means that
    \begin{align*}
        N_{\alpha}(g) = \sup A \leq 1,
    \end{align*}
    as the supremum must be the least upper bound.

    % Suppose, for the sake of contradiction, \(N_{\alpha}(g) > 1\), i.e., \(N_{\alpha}(g) = 1 + \delta\) for some \(\delta > 0\).
    % This means that there exist \(x_0,y_0 \in X, x_0 \neq y_0\) s.t.,
    % \begin{align*}
    %     \frac{\abs{g(x_0)-g(y_0)}}{d(x_0, y_0)^{\alpha}} > 1 + \delta/2,
    % \end{align*}
    % as \(1 + \delta/2 < N_{\alpha}(g)\) so \(1 + \delta/2\) is not an upper bound.

    % Set \(\lambda = (\delta/2)d(x_0,y_0)^{\alpha}\).
    % Note that since \(x_0 \neq y_0\), \(d(x_0, y_0) > 0\).
    % Also, as \(\alpha > 0\), \(d(x_0, y_0)^{\alpha} > 0\).
    % Thus, \(\lambda > 0\).
    % And by rearranging the above inequality,
    % \begin{align*}
    %     \abs{g(x_0)-g(y_0)} > d(x_0, y_0)^{\alpha} + \lambda.
    % \end{align*}
    % And since \(g_n \to g\) uniformly, there exists a natural \(N\) s.t. for all \(x \in X\),
    % \begin{align*}
    %     \abs{g_N(x) - g(x)} < \frac{\lambda}{2}.
    % \end{align*}
    % This gives in particular that,
    % \begin{align*}
    %     g(x_0)- \frac{\lambda}{2} < g_N(x_0) < g(x_0) + \frac{\lambda}{2} \text{  and  } g(y_0) - \frac{\lambda}{2} < g_N(y_0) < g(y_0)+ \frac{\lambda}{2}.
    % \end{align*}
    % This gives that

    \textbf{Bounded:}
    It's enough to show that \(\Gamma\) can be covered in an open neighborhood in the metric space \((\cC(X), \norm{\cdot})\).
    Let \(\opb{0}{2}\) be the open ball centered at the zero function with a radius 2.
    We claim that 
    \begin{align*}
        \opb{0}{2} \supseteq \Gamma.
    \end{align*}
    To see this, pick any \(f \in \Gamma\), then \(\norm{f} = \norm{f - 0} \leq 1 < 2\).
    Therefore, \(f \in \opb{0}{2}\).

    \textbf{Equicontinous:}
    

\end{fproof}

\begin{fproof}[1(b)]

\end{fproof}
\newpage

% Problem 2
\begin{fproof}[2]

\end{fproof}
\newpage

% Problem 3
\begin{fproof}[3(a)]
  
\end{fproof}

\begin{fproof}[3(b)]
  
\end{fproof}
\newpage

% Problem 4
\begin{fproof}[4]

\end{fproof}
\newpage

% Problem 5
\begin{fproof}[5]

\end{fproof}
\end{document}