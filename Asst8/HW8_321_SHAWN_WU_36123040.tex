\documentclass[12pt]{article}
\usepackage{mymarco} % my marco file
\usepackage{parskip} % no auto indent
\usepackage[margin=1.0in]{geometry}
\usepackage{fancyhdr}
\pagestyle{fancy}
\setlength{\headheight}{15pt}
\rhead{36123040 Shawn Wu} % Define header
\lhead{MATH 321 HW 08} % Put assignment #
\cfoot{\thepage}

\begin{document}

% Problem 1
\begin{fproof}[1(a)]
 First note that for any fixed \(a > 0\) and \(x \neq 0\).
 \begin{align*}
    \frac{a^x - 1}{x} = \frac{e^{x \log a} - e^{x \log 1}}{x} = \brac{\frac{e^{xt}}{x}}_{t= \log 1}^{t= \log a} = \int_{\log 1}^{\log a} e^{x t} \dt.
 \end{align*}
 Now consider any real sequence \((x_n) \subseteq \R \setminus \{0\}\) s.t. \(x_n \to 0\).
 Define \(f : I \to \R\), where \(I\) is the closed interval between \(\log a\) and \(\log 1(=0)\), as
 \begin{align*}
    f_n(t) = e^{x_n t}.
 \end{align*}
 We claim that \(f_n(t)\) converges uniformly to \(f(t) = 1\) on \(I\).
 To see this, first note that for each \(n\), \(f_n\) is monotone on \(I\).
 Therefore,
 \begin{align*}
    M_n &= \sup_{t \in I} \abs{f_n(t) - f(t)} \tag{by the def. of \(M_n\)}\\
    &= \sup_{t \in I} \abs{f_n(t) - f_n(0)} \tag{since \(f_n(0) = f(t), \forall n,t\) }\\
    &= \sup_{t \in I} \abs{f_n(\log a) - f_n(0)} \tag{since \(f_n\) monotone on \(I\)}\\
    &=\sup_{t \in I} \abs{f_n(\log a) - 1} \tag{since \(f_n(0) = 1\)}.
 \end{align*}
 Therefore, \(M_n \to 0\) since \(f_n(\log a) \to 1\) as \(n \to \infty\).
 This concludes the proof that \(f_n(t)\) converges uniformly to \(1\).
 Hence by Rudin Theorem 7.16,
 \begin{align*}
    \lim_{n\to \infty} \int_{\log 1}^{\log a} f_n(t) \dt = \int_{\log 1}^{\log a} f(t) \dt, \text{ i.e.,}
    \lim_{n\to \infty} \int_{\log 1}^{\log a} e^{x_n t} \dt = \int_{\log 1}^{\log a} 1 \dt = \log a.
 \end{align*}
Since \((x_n)\) is an arbitrary sequence in \(\R \setminus \{0\}\) that converges to \(0\), by Rudin Theorem 4.2,
\begin{align*}
    \lim_{x \to 0} \int_{\log 1}^{\log a} e^{xt} 
    dt = \log a.
\end{align*}
Combining with the previous results, we can thus conclude that,
\begin{align*}
    \lim_{x \to 0} \frac{a^x - 1}{x} = \log a
\end{align*}
\end{fproof}

\begin{fproof}[1(b)]
 First note that by HW7(a), if \(x \neq 0\) is small enough, say \(|x| < 1\),
 \begin{align*}
    \log(1+x)= x - \frac{x^2}{2} + \frac{x^3}{3} - \frac{x^4}{4} + \cdots
 \end{align*}
 This means that for a fixed \(x \neq 0\) small enough,
 \begin{align*}
    \frac{1}{x^2}\brac{\log(1+x) - x} 
    &= \frac{1}{x^2} \brac{-\frac{x^2}{2} + \frac{x^3}{3} - \frac{x^4}{4} + \frac{x^5}{5} - \cdots}\\
    & = -\frac{1}{2} + \frac{x}{3} - \frac{x^2}{4} + \frac{x^3}{5} - \frac{x^4}{6} + \cdots \tag{since \(\sum_{k=2}^{\infty} (-1)^{k-1} \frac{x^k}{k}\) converges}\\
    &=-\frac{1}{2} + \sum_{k=1}^{\infty} (-1)^{k-1} \frac{x^{k}}{k+2}.
 \end{align*}
 Note that since
 \begin{align*}
   \limsup_{k \to \infty} \sqrt[k]{\frac{1}{k+2}} = 1,
 \end{align*}
 the series \(\sum_{k=1}^{\infty} (-1)^{k-1} \frac{x^{k}}{k+2}\) has raidus of convergence of \(1\).
 This means that on \((-1,1)\), the function \(f(x) = \sum_{k=1}^{\infty} (-1)^{k-1} \frac{x^{k}}{k+2}\) is continuous by Rudin Theorem 8.1.
 Therefore,
 \begin{align*}
   \lim_{x \to 0} \frac{\log(1 + x)-x}{x^2} 
   & = \lim_{x \to 0} -\frac{1}{2} + \sum_{k=1}^{\infty} (-1)^{k-1} \frac{x^{k}}{k+2} \tag{since they agree on \(|x| < 1\)}\\
   & = -\frac{1}{2} + \lim_{x \to 0} \sum_{k=1}^{\infty} (-1)^{k-1} \frac{x^{k}}{k+2} \tag{since the series converges}\\
   &= -\frac{1}{2} + f(0) \tag{since \(f\) is continous at \(0\)}\\
   & = -\frac{1}{2}. \tag{since \(f(0)=0\)}
 \end{align*}
\end{fproof}

\begin{fproof}[1(c)]
 First note that with the results in Rudin Theorem 3.31, it's clear that for any sequence \((a_n) \subseteq \R\) where \(a_n \to \infty\),
 \begin{align*}
   \lim_{n \to \infty} \pare{1 + \frac{1}{a_n}}^{a_n} = e.
 \end{align*}
 We thus fix \(x \in \R\).
 If \(x = 0\), then 
 \begin{align*}
   \lim_{n \to \infty} \pare{1 + \frac{0}{n}}^n = \lim_{n \to \infty} 1 = 1 = e^0.
 \end{align*}
 If \(x \neq 0\), then
 \begin{align*}
   \lim_{n \to \infty} \pare{1 + \frac{x}{n}}^n = \brac{\pare{1 + \frac{1}{n/x}}^{n/x}}^x = e^x,
 \end{align*}
 since as \(n \to \infty\), \(n/x \to \infty\).
\end{fproof}
\newpage

% Problem 2
\begin{fproof}[2(a)]
 
\end{fproof}

\begin{fproof}[2(b)]
 
\end{fproof}
\newpage

% Problem 3
\begin{fproof}[3(a)]
 
\end{fproof}

\begin{fproof}[3(b)]
 
\end{fproof}

\begin{fproof}[3(c)]
 
\end{fproof}
\newpage

% Problem 4
\begin{fproof}[4(a)]
 
\end{fproof}

\begin{fproof}[4(b)]
 
\end{fproof}

\begin{fproof}[4(c)]
 
\end{fproof}
\end{document}