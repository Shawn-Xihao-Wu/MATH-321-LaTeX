\documentclass[12pt]{article}
\usepackage{mymarco} % my marco file
\usepackage{parskip} % no auto indent
\usepackage[margin=1.0in]{geometry}
\usepackage{fancyhdr}
\pagestyle{fancy}
\setlength{\headheight}{15pt}
\rhead{36123040 Shawn Wu} % Define header
\lhead{MATH 321 HW 08} % Put assignment #
\cfoot{\thepage}

\begin{document}

% Problem 1
\begin{fproof}[1(a)]
 First note that for any fixed \(a > 0\) and \(x \neq 0\).
 \begin{align*}
    \frac{a^x - 1}{x} = \frac{e^{x \log a} - e^{x \log 1}}{x} = \brac{\frac{e^{xt}}{x}}_{t= \log 1}^{t= \log a} = \int_{\log 1}^{\log a} e^{x t} \dt.
 \end{align*}
 Now consider any real sequence \((x_n) \subseteq \R \setminus \{0\}\) s.t. \(x_n \to 0\).
 Define \(f : I \to \R\), where \(I\) is the closed interval between \(\log a\) and \(\log 1(=0)\), as
 \begin{align*}
    f_n(t) = e^{x_n t}.
 \end{align*}
 We claim that \(f_n(t)\) converges uniformly to \(f(t) = 1\) on \(I\).
 To see this, first note that for each \(n\), \(f_n\) is monotone on \(I\).
 Therefore,
 \begin{align*}
    M_n &= \sup_{t \in I} \abs{f_n(t) - f(t)} \tag{by the def. of \(M_n\)}\\
    &= \sup_{t \in I} \abs{f_n(t) - f_n(0)} \tag{since \(f_n(0) = f(t), \forall n,t\) }\\
    &= \sup_{t \in I} \abs{f_n(\log a) - f_n(0)} \tag{since \(f_n\) monotone on \(I\)}\\
    &=\sup_{t \in I} \abs{f_n(\log a) - 1} \tag{since \(f_n(0) = 1\)}.
 \end{align*}
 Therefore, \(M_n \to 0\) since \(f_n(\log a) \to 1\) as \(n \to \infty\).
 This concludes the proof that \(f_n(t)\) converges uniformly to \(1\).
 Hence by Rudin Theorem 7.16,
 \begin{align*}
    \lim_{n\to \infty} \int_{\log 1}^{\log a} f_n(t) \dt = \int_{\log 1}^{\log a} f(t) \dt, \text{ i.e.,}
    \lim_{n\to \infty} \int_{\log 1}^{\log a} e^{x_n t} \dt = \int_{\log 1}^{\log a} 1 \dt = \log a.
 \end{align*}
Since \((x_n)\) is an arbitrary sequence in \(\R \setminus \{0\}\) that converges to \(0\), by Rudin Theorem 4.2,
\begin{align*}
    \lim_{x \to 0} \int_{\log 1}^{\log a} e^{xt} 
    dt = \log a.
\end{align*}
Combining with the previous results, we can thus conclude that,
\begin{align*}
    \lim_{x \to 0} \frac{a^x - 1}{x} = \log a
\end{align*}
\end{fproof}

\begin{fproof}[1(b)]
 First note that by HW7(a), if \(x \neq 0\) is small enough, say \(|x| < 1\),
 \begin{align*}
    \log(1+x)= x - \frac{x^2}{2} + \frac{x^3}{3} - \frac{x^4}{4} + \cdots
 \end{align*}
 This means that for a fixed \(x \neq 0\) small enough,
 \begin{align*}
    \frac{1}{x^2}\brac{\log(1+x) - x} 
    &= \frac{1}{x^2} \brac{-\frac{x^2}{2} + \frac{x^3}{3} - \frac{x^4}{4} + \frac{x^5}{5} - \cdots}\\
    & = -\frac{1}{2} + \frac{x}{3} - \frac{x^2}{4} + \frac{x^3}{5} - \frac{x^4}{6} + \cdots \tag{since \(\sum_{k=2}^{\infty} (-1)^{k-1} \frac{x^k}{k}\) converges}\\
    &=-\frac{1}{2} + \sum_{k=1}^{\infty} (-1)^{k-1} \frac{x^{k}}{k+2}.
 \end{align*}
 Note that since
 \begin{align*}
   \limsup_{k \to \infty} \sqrt[k]{\frac{1}{k+2}} = 1,
 \end{align*}
 the series \(\sum_{k=1}^{\infty} (-1)^{k-1} \frac{x^{k}}{k+2}\) has raidus of convergence of \(1\).
 This means that on \((-1,1)\), the function \(f(x) = \sum_{k=1}^{\infty} (-1)^{k-1} \frac{x^{k}}{k+2}\) is continuous by Rudin Theorem 8.1.
 Therefore,
 \begin{align*}
   \lim_{x \to 0} \frac{\log(1 + x)-x}{x^2} 
   & = \lim_{x \to 0} -\frac{1}{2} + \sum_{k=1}^{\infty} (-1)^{k-1} \frac{x^{k}}{k+2} \tag{since they agree on \(|x| < 1\)}\\
   & = -\frac{1}{2} + \lim_{x \to 0} \sum_{k=1}^{\infty} (-1)^{k-1} \frac{x^{k}}{k+2} \tag{since the series converges}\\
   &= -\frac{1}{2} + f(0) \tag{since \(f\) is continous at \(0\)}\\
   & = -\frac{1}{2}. \tag{since \(f(0)=0\)}
 \end{align*}
\end{fproof}

\begin{fproof}[1(c)]
 First note that with the results in Rudin Theorem 3.31, it's clear that for any sequence \((a_n) \subseteq \R\) where \(a_n \to \infty\),
 \begin{align*}
   \lim_{n \to \infty} \pare{1 + \frac{1}{a_n}}^{a_n} = e.
 \end{align*}
 We thus fix \(x \in \R\).
 If \(x = 0\), then 
 \begin{align*}
   \lim_{n \to \infty} \pare{1 + \frac{0}{n}}^n = \lim_{n \to \infty} 1 = 1 = e^0.
 \end{align*}
 If \(x \neq 0\), then
 \begin{align*}
   \lim_{n \to \infty} \pare{1 + \frac{x}{n}}^n = \brac{\pare{1 + \frac{1}{n/x}}^{n/x}}^x = e^x,
 \end{align*}
 since as \(n \to \infty\), \(n/x \to \infty\).
\end{fproof}
\newpage

% Problem 2
\begin{fproof}[2(a)]
 
\end{fproof}

\begin{fproof}[2(b)]
 
\end{fproof}
\newpage

% Problem 3
\begin{fproof}[3(a)]
 Let's first compute each \(c_m\).
 By Rudin (62),
 \begin{align*}
   c_m 
   = \frac{1}{2\pi} \int_{-\pi}^{\pi} f(x) e^{-imx} \dx.
 \end{align*}
 If \(m = 0\), then with the periodicity of the \(f\),
 \begin{align*}
   c_m = \frac{1}{2 \pi} \int_{-\pi}^{\pi} f(x) \dx = 0.
 \end{align*}
 If \(m \neq 0\), then with integration by parts,
 \begin{align*}
   c_m 
   &= \frac{1}{2\pi} \int_{-\pi}^{0} \frac{-\pi-x}{2} e^{-imx} \dx + \frac{1}{2\pi} \int_{-\pi}^{0} \frac{\pi-x}{2} e^{-imx} \dx\\
   & = \frac{-i\pi m +e^{i\pi m} - 1}{4 \pi m^2} + \frac{-i\pi m - e^{-i\pi m} + 1}{4 \pi m^2}.
 \end{align*}
 Therefore, when \(m\) is even,
 \begin{align*}
   c_m = \frac{-i \pi m + 1 - 1}{4 \pi m^2} + \frac{-i \pi m - 1 + 1}{4 \pi m^2} = \frac{-i}{2m},
 \end{align*}
 and similarly, when \(m\) is odd,
 \begin{align*}
   c_m = \frac{-i \pi m + (-1) - 1}{4 \pi m^2} + \frac{-i \pi m - (-1) + 1}{4 \pi m^2} = \frac{-i}{2m}.
 \end{align*}
 \begin{center}
   \(\ast~\ast~\ast\)
 \end{center}
 Now note that for \(m \neq 0\),
 \begin{align*}
   c_m e^{imx}
   &= -\frac{1}{2m}e^{i(mx + \pi/2)}\\
   &= -\frac{1}{2m} \pare{\cos(mx + \pi/2) + i \sin(mx + \pi/2)}\\
   &= \frac{1}{2m} \pare{\sin(mx) - i \sin(mx + \pi/2)}\\
   &= \frac{1}{2m} \pare{\sin(mx) - i\cos(mx)}.
 \end{align*}
 And similarly,
 \begin{align*}
   c_{-m}e^{-imx}
   & = \frac{1}{2(-m)}\pare{\sin(-mx) - i\sin(-mx + \pi/2)}\\
   &=\frac{1}{2m} \pare{\sin(mx) + i\sin(-mx + \pi/2)}\\
   &= \frac{1}{2m} \pare{\sin(mx) + i\sin(-(mx - \pi/2))}\\
   &=\frac{1}{2m} \pare{\sin(mx) - i\sin(mx - \pi/2)}\\
   &=\frac{1}{2m} \pare{\sin(mx) + i\cos(mx)}.
 \end{align*}
 This means that,
 \begin{align*}
   c_m e^{imx} + c_{-m}e^{-imx} = \frac{1}{m} \sin(mx).
 \end{align*}
 Therefore,
 \begin{align*}
   \sum_{m = -\infty}^{\infty} c_m e^{imx} = \sum_{n=0}^{\infty} \frac{1}{n} \sin(nx).
 \end{align*}
 Also since \(f(x)\) is differentiable on \([-\pi, 0)\) and \((0, \pi]\), \(f(x)\) is thus Lipschitz continuous on there as well.
 Therefore, by Rudin Theorem 8.14, on those intervals, \(f(x)\)'s Fourier series converge to \(f(x)\), which means that on those intervals,
 \begin{align*}
   f(x) = \sum_{n=0}^{\infty} \frac{1}{n} \sin(nx).
 \end{align*}

\end{fproof}

\begin{fproof}[3(b)]
 
\end{fproof}

\begin{fproof}[3(c)]
 
\end{fproof}
\newpage

% Problem 4
\begin{fproof}[4(a)]
 First notice that for any \(x \in (0, \pi/2)\),\((\pi - x, \pi + x) \subseteq (\pi/2, 3\pi/2)\).
 This means that \(\pi/2 < s < 3\pi/2 \), so \(\pi/4 < s/2 < 3\pi/4\).
 Therefore, \(\sin(s/2) \neq 0\) on any \((\pi - x, \pi + x)\).
 This means that the integral of \(D_N(s)\) over \((\pi - x, \pi + x)\) is well-defined.

 Now we fix \(x \in (0, \pi/2)\).
 Note that
 \begin{align*}
   \int_{\pi - x}^{\pi+x} D_N(s) 
   &= \int_{\pi - x}^{\pi+x} \frac{\sin(Ns + s/2)}{s/2} \ds\\
   &=\int_{\pi - x}^{\pi+x}\frac{\sin(Ns)\cos(s/2)+\cos(Ns)\sin(s/2)}{\sin(s/2)} \ds\\
   &=\int_{\pi - x}^{\pi+x} \sin(Ns) \cot(s/2) \ds + \int_{\pi - x}^{\pi+x} \cos(Ns) \ds.
 \end{align*}
 Now, define \(g: [-\pi, \pi] \to \R\) as
 \[
 g(s) = \begin{cases}
   \cot(s/2) & \text{if } s \in (\pi-x, \pi+x),\\
   0 & \text{otherwise}.
 \end{cases}
 \]
 Note that \(g(s)\in \cR[-\pi, \pi]\), since \(\cot(s/2)\) is continous on \(\pi-x, \pi+x\), which means that \(g(s)\) has finitely many discontinuities on \([-\pi, \pi]\).
 Therefore, what we've learned in class,
 \begin{align*}
   \lim_{N\to \infty} \int_{-\pi}^{\pi} \sin(Ns) g(s) \ds = 0.
 \end{align*}
 Noice also that for each \(N\),
 \begin{align*}
   \int_{-\pi}^{\pi} \sin(Ns) g(s) \ds = \int_{\pi-x}^{\pi} \sin(Ns) \cot(s/2) \ds.
 \end{align*}
 Thus, 
 \begin{align*}
   \lim_{N\to \infty} \int_{\pi-x}^{\pi} \sin(Ns) \cot(s/2) \ds = 0.
 \end{align*}
 And with a reasoning along with the peridocity of \(\sin\) and \(\cot\), we also get that
 \begin{align*}
   \lim_{N\to \infty} \int_{\pi}^{\pi+x} \sin(Ns) \cot(s/2) \ds = 0.
 \end{align*}
 Therefore, 
 \begin{align*}
   &\lim_{N \to \infty} \int_{\pi-x}^{\pi+x} \sin(Ns) \cot(s/2) \ds \\
   =& \lim_{N \to \infty} \pare{\int_{\pi-x}^{\pi} \sin(Ns) \cot(s/2) \ds + \int_{\pi}^{\pi+x} \sin(Ns) \cot(s/2) \ds}\\
   =&\lim_{N\to \infty} \int_{\pi-x}^{\pi} \sin(Ns) \cot(s/2) \ds + \lim_{n\to \infty} \int_{\pi}^{\pi+x} \sin(Ns) \cot(s/2) \\
   =& 0 +0\\
   =& 0.
 \end{align*}
 With a similar reasoning, we can get that
 \begin{align*}
   \lim_{N \to \infty}\int_{\pi - x}^{\pi+x} \cos(Ns) \ds = 0.
 \end{align*}
 Therefore,
 \begin{align*}
   \lim_{N\to\infty} \int_{\pi - x}^{\pi+x} D_n(s) \ds, \text{ which gives that } \lim_{N \to \infty} r_N(x) = 0.
 \end{align*}
\end{fproof}

\begin{fproof}[4(b)]
 
\end{fproof}

\begin{fproof}[4(c)]
 
\end{fproof}
\end{document}